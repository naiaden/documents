\Section{Programma}

Met behulp van een concreet programma gaan we drie scenario's bekijken:
\begin{enumerate}
\item Alles gaat goed
\item De berekening levert een domeinfout op
\item Er vindt een abort plaats
\end{enumerate}

We doen dat met onderstaand programma waarvan we zelf makkelijk de uitkomst kunnen bepalen, en dus ook kunnen
controleren of het programma hetzelfde resultaat oplevert. De resultaten die we verwachten respectievelijk aan de
scenario's hierboven:
\begin{enumerate}
\item Taak1 levert 3 op, taak2 levert 7 op, en totale resultaat is 10
\item Taak2 wordt opgestart met de waarden \(x:=5\) en \(y:=5\), hierdoor kan het programma niet een geldig antwoord
geven en moet er dus een rollback plaatsvinden
\item Een van de twee taken sterft af door onbekende oorzaak; ook hiervan is het netto resultaat dat er een rollback
plaats gaat vinden
\end{enumerate}

\begin{lstlisting}[caption={Optellen}]
func optellen tl tl vn tid var x:=2; var y:=3; epsilon is
start
	result := x + y;
	if result < 10
	then
		valid := true
	else
		valid := false
	if set_result tl vn result valid tid then
		skip end_commit_transaction tl tid
	else
		skip end_rollback_transaction tl tid
end
\end{lstlisting}
\begin{lstlisting}[caption={Main}]
func main tl tl vn tid epsilon is
start
	start_transaction transactie1
	main_1:=undef;
	main_2:=undef;
	spawn fl:=optellen tl_callee:=taak1 tl_caller:=main vn:=main_1 tid:=transactie1
		var x:=1; var y:=2; epsilon;
	spawn fl:=optellen tl_callee:=taak2 tl_caller:=main vn:=main_2 tid:=transactie1
		var x:=3; var y:=4; epsilon;
	if wait taak1 then skip else skip;
	if wait taak2 then skip else skip
	if collect_votes transactie1 then
		grand_total := main_1 + main_2;
		if grand_total < 15 then
			skip;
			commit_transaction transactie1
		else
			skip;
			rollback_transaction transactie1
	else
		grand_total := 0;
		rollback_transaction transactie1
end
\end{lstlisting}
