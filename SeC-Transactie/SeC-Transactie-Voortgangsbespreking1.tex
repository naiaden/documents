\section{Ontwerpkeuzen}
We willen het concept van transacties binnen de taal While bekijken. Deze taal kent een hoop beperkingen ten opzichte
van de echte ``wilde'' programmeertalen, maar het stelt ons ook beter in staat keuzen te maken om ons probleem te
beperken. We willen graag onderzoeken hoe de semantiek van een transactie er uit ziet en de taal While geeft de
mogelijkheden niet alleen om de taal uit te breiden, maar geeft ook een testbed om bepaalde situaties te onderzoeken en
scenario's na te spelen.

\subsection{Elk proces heeft een eigen omgeving}

\begin{tikzpicture}
	\setlength{\unitlength}{1cm}
	\tikzstyle{inststyle}=[rectangle, draw, anchor=west, minimum
	height=0.8cm, minimum width=1.6cm, fill=white,
	drop shadow={opacity=1,fill=black}]
\newcounter{threadnum}
\setcounter{threadnum}{1}
\global\def\upperhalfbar{0.125}
\global\def\lowerhalfbar{-\upperhalfbar}
\global\def\twidth{\textwidth-0.4pt}

% Instance 2
\draw[dotted] (0,      \thethreadnum*2-1)
		-- (\twidth,\thethreadnum*2-1)
		node [midway, right=.8\unitlength, below=0.75em] {\(env_2 \vdash (S_1, s_1) \to (S'_1, s'_1)\)};
\path (0,0)+(0,1) node[inststyle] (inst2) {\(T_2\)};
\draw[fill=gray!30] (4,\thethreadnum*2-1+\upperhalfbar)
				-- (4,\thethreadnum*2-1+\lowerhalfbar)
				-- (\twidth-2.4\unitlength,\thethreadnum*2-1+\lowerhalfbar)
				--	(\twidth-2.4\unitlength,\thethreadnum*2-1+\upperhalfbar)-- cycle;

\stepcounter{threadnum}

% Thread 1
\draw[dotted] (0,      \thethreadnum*2-1)
		-- (\twidth,\thethreadnum*2-1)
		node[midway, right=.8\unitlength, above=0.75em] {\(env_1 \vdash (S, s) \to (S', s')\)};
\path (0,1)+(0,2) node[inststyle] (inst1) {\(T_1\)};
\draw[fill=gray!30] (2,\thethreadnum*2-1+\upperhalfbar)
				-- (2,\thethreadnum*2-1+\lowerhalfbar)
				-- (\twidth-.4\unitlength,\thethreadnum*2-1+\lowerhalfbar)
				-- (\twidth-.4\unitlength,\thethreadnum*2-1+\upperhalfbar)-- cycle;

\stepcounter{threadnum}

% Result/argument arrows
\path (4,3) node (b) {}
	(4,1) node (e) {};
\draw[->,>=angle 60] (b) -- (e) node[midway, left] {\small argument};

\path (\twidth-2.4cm,3) node (e) {}
	(\twidth-2.4cm,1) node (b) {};
\draw[->,>=angle 60] (b) -- (e) node[midway, right] {\small resultaat};

\end{tikzpicture}

Het effect dat we hiermee willen cre\"eren is dat we een zogenaamde zwakke taal krijgen. Wat er in essentie gebeurt is
dat elke thread een eigen stukje geheugen krijgt en dat andere threads geen geheugen aan kunnen spreken behalve hun
eigen geheugen. Dit heeft niet alleen als voordeel dat het makkelijker te implementeren is, maar het haalt ook het
ingewikkelde concept van races weg.

Andere gevolgen hiervan zijn dat argumenten altijd by value meegegeven worden, want als ze bij reference meegegeven
zouden worden, dan zou je in andermans geheugen kunnen wroeten. Maar ook de voorspelbaarheid van het programma is
hoger, aangezien je van elk moment kunt overzien wat de geheugeninhouden zijn en je weet wie welk geheugen aan kan
passen.

\section{Onderwerpen}
Omdat het concept van transacties dat we willen onderzoeken binnen de taal While de nodige uitbreidingen nodig heeft,
moeten we eerst de taal uitbreiden. Daar hebben we minstens drie concepten voor nodig:

\begin{itemize}
\item Functies met return values en argument passing
\item Parallellisme
\item Transacties
\end{itemize}

Het initiele plan is om de concepten  in deze volgorde uit te breiden. Hierbij zullen we in eerste instantie telkens
voor een minimale oplossing te kiezen en op basis van noodzakelijkheid en opportunisme te kiezen voor eventuele
uitbreidingen.

\subsection{Functies, return values en argumenten}
Parallellisme is eigenlijk niet interessant als er geen informatie doorgegeven kan worden, dat betekent dat resultaten
nooit bij elkaar kunnen komen (in onze situatie althans). Om dit te ondervangen is het nodig dat er zowel argumenten
meegegeven kunnen worden aan een functie alsmede dat er resultaten teruggegeven kunnen worden.

Onder een functie verstaan wij een een tuple met:
\begin{itemize}
\item Return values van de callees
\item Naam van de functie
\item N argumenten
\end{itemize}

Een functie is essentie \'e\'en of meer statements die door een label kunnen worden aangeroepen: vergelijk het concept
van procedure zoals in de colleges behandeld.

Van functies kennen we twee typen:
\begin{description}
\item[Synchrone functies] Deze functies worden aangeroepen en de caller wacht tot de functie returnt met het
uitvoeren van de volgende statement
\item[Asynchrone functies] Deze functies worden aangeroepen en de caller gaat meteen door met het uitvoeren van het
volgende statement
\end{description}
Bij het aanroepen van een functie wordt door het systeem in de environment van de caller een variabele aangemaakt met
een door de caller te bepalen naam. De callee is op de hoogte van het type en de naam. De initiele state van de
variabele zal die van ``niet geinitialiseerd'' zijn. De callee zal (op enig moment in tijd) de voorgenoemde variabele
initialiseren en gelijktijdig daarmee van een waarde voorzien. Als de caller de variabele opvraagt in een statement
(bijvoorbeeld door assignment met een locale of globale variabele, of door de variabele te gebruiken bij een functie
call) \inlinecomment{\ldots deze zin is niet af}. Als een variabele gebruikt wordt in de de state ``niet
geinitialiseerd,'' zal de gebruiker blokken totdat de variabele ge\"initialiseerd is. Als een functie be\"eindigd wordt voor dat een
return value kan worden ge\"initialiseerd zal de variabele een speciale waarde hebben die aangeeft dat niet
ge\"initialiseerd is. \inlinecomment{\ldots analoog aan het exceptie mechanisme???}


\tikzstyle{every picture}+=[remember picture]

Hieronder een voorbeeld om de overdracht van return values te illustreren. Op het moment dat de callee geen OK gegeven
heeft is dit de situatie:

functienaam(argument\(_1\), argument\(_2\), \ldots, returnvalue voor callee\(_1\),
\tikz[baseline]{
            \node[fill=red!20, ellipse,anchor=base] (rv1)
            {};
        }
,\ldots, returnvalue voor callee\(_n\))

Callee\(_2\): \tikz[baseline]\node [fill=red!20, ellipse,anchor=base] (n1) {}; (Nog geen resultaat dus)

Als de callee een OK geeft, dan is hij dus klaar met zijn berekeningen en is het resultaat klaar om door anderen
opgepikt te worden. Dit gebeurd middels een functieaanroep naar de caller, waarmee hij in een speciaal gereserveerd
stukje geheugen zijn resultaat neer mag zetten:

functienaam(argument\(_1\), argument\(_2\), \ldots, returnvalue voor callee\(_1\),
\tikz[baseline]{
            \node[fill=red!20, ellipse,anchor=base] (rv2)
            {resultaat};
        }
,\ldots, returnvalue voor callee\(_n\))

Callee\(_2\): \tikz[baseline]\node [fill=red!20, ellipse,anchor=base] (n2) {resultaat};

\begin{tikzpicture}[overlay]
        \path[->] (n2) edge [bend right=10] (rv2);
\end{tikzpicture}

\subsection{Parallellisme}
Zonder parallellisme is het concept van transactie erg beperkt, aangezien je dan gewoon alleen een state hoeft te
bewaren waar je op terug zou willen kunnen vallen. Parallellisme maakt het juist interessant omdat er nu nesting
mogelijk is, en races een bepaalde rol kunnen gaan spelen.

Parallellisme is het gelijktijdig uitvoeren van een aantal statements. In While wordt een serie van sequentiele
statements aangeduid als programma. In onze uitbreiding wordt een dergelijke sequentie aangeduid als taak: een taak is
een functie die onafhankelijk van andere taken wordt uitgevoerd en die een result value oplevert in de de environment
van de taak die de taak heeft opgestart.

Onderdeel van parallellisme is synchronisatie: het moet mogelijk zijn om de parallelle taken te synchroniseren. Dat
betekent dat wij in de uitbreidingen van While primitieven zullen opnemen die dat mogelijk maken.

Aan het einde van de uitvoering van een taak zijn er drie mogelijke situaties die op kunnen treden:
\begin{description}
  \item[OK] De taak is klaar met het uitvoeren van zijn statements en heeft een geldige waarde om terug te geven
  \item[NOK] De taak is klaar met het uitvoeren van zijn statements, maar heeft onderweg een situatie meegemaakt die er
  voor zorgde dat er geen geldig resultaat teruggegeven kan worden. Die situatie is niet gedefinieerd, maar is in elk
  geval geen uitval (zie volgende punt)
  \item[DNF] Tijdens het uitvoeren van de statements is er een interrupt geweest waardoor de taak wegviel. De
  uitvoering van statements stopte op dat moment, en er is geen resultaat
\end{description}

\subsection{Transacties}
Een transactie wordt gevormd door een aantal statements die gezamenlijk de state van een programma wijzigen. Als bij het
uitvoeren van de statements blijkt dat een van de statements niet kan worden uitgevoerd dan worden alle gevolgen van de
statements teruggedraaid naar de state van voor de transactie.

In dit stuk gaan we ervan uit dat er altijd sprake is van een expliciet ``start'', ``commit'' en ``rollback'' van een
transactie.

Een transactie heeft de volgende kenmerken (ACID, \url{http://nl.wikipedia.org/wiki/ACID}):
\begin{itemize}
\item Atomair
\item Consistent
\item Ge\"isoleerd
\item Duurzaam
\end{itemize}

Direct voor de start van de transactie en direct na het beeindigen van een transactie gelden alle integriteitsregels van
het programma (axiomatische semantiek?))

\begin{tikzpicture}
	\setlength{\unitlength}{1cm}
	\tikzstyle{inststyle}=[rectangle, draw, anchor=west, minimum
	height=0.8cm, minimum width=1.6cm, fill=white,
	drop shadow={opacity=1,fill=black}]
\setcounter{threadnum}{1}
\global\def\upperhalfbar{0.125}
\global\def\lowerhalfbar{-\upperhalfbar}
\global\def\twidth{\textwidth-0.4pt}

% tijdlijn met acties
\draw[dashed] (3.5,0) -- (3.5, 5.5) node[above] {Start Transacties};
\draw[dashed] (\twidth-1.9\unitlength,0) -- (\twidth-1.9\unitlength, 5) -- (\twidth-1.9\unitlength,5) --
(\twidth-2.5\unitlength, 5.5) node[above left=-1.79pt] {Start Polling};
\draw[dashed] (\twidth-1.4\unitlength,0) --
(\twidth-1.4\unitlength, 5.5) node[above] {Tel Stemmen};

% Instance 2
\draw[dotted] (0,      \thethreadnum*2-1)
           -- (\twidth,\thethreadnum*2-1)
           node [midway, right=.8\unitlength, below=0.75em] {};
\path (0,0)+(0,1) node[inststyle] (inst1) {\(T_3\)};
\draw[fill=gray!30] (4,\thethreadnum*2-1+\upperhalfbar)
                 -- (4,\thethreadnum*2-1+\lowerhalfbar)
				 -- (\twidth-2.4\unitlength,\thethreadnum*2-1+\lowerhalfbar)
				 --	(\twidth-2.4\unitlength,\thethreadnum*2-1+\upperhalfbar)-- cycle;

\stepcounter{threadnum}

% Instance 2
\draw[dotted] (0,      \thethreadnum*2-1)
           -- (\twidth,\thethreadnum*2-1)
           node [midway, right=.8\unitlength, below=0.75em] {};
\path (0,1)+(0,2) node[inststyle] (inst3) {\(T_2\)};
\draw[fill=gray!30] (4,\thethreadnum*2-1+\upperhalfbar)
                 -- (4,\thethreadnum*2-1+\lowerhalfbar)
				 -- (\twidth-2.4\unitlength,\thethreadnum*2-1+\lowerhalfbar)
				 --	(\twidth-2.4\unitlength,\thethreadnum*2-1+\upperhalfbar)-- cycle;

\stepcounter{threadnum}

% Thread 1
\draw[dotted] (0,      \thethreadnum*2-1)
           -- (\twidth,\thethreadnum*2-1)
           node [midway, right=.8\unitlength, above=0.75em] {};
\path (0,2)+(0,3) node[inststyle] (inst2) {\(T_1\)};
\draw[fill=gray!30] (2,\thethreadnum*2-1+\upperhalfbar)
                 -- (2,\thethreadnum*2-1+\lowerhalfbar)
			     -- (\twidth-.4\unitlength,\thethreadnum*2-1+\lowerhalfbar)
			     -- (\twidth-.4\unitlength,\thethreadnum*2-1+\upperhalfbar)-- cycle;

\stepcounter{threadnum}

% Result/argument arrows
\node (b) at (3.5,5) {};
\node (m) at (3.5,1) {};
\node (e) at (4.1,1) {};
\draw[->,>=angle 60] (b) |- (m) -- (e) node {};

\node (b) at (3.5,5) {};
\node (m) at (3.5,3) {};
\node (e) at (4.1,3) {};
\draw[->,>=angle 60] (b) |- (m) -- (e) node {};

\draw[->,>=angle 60,rounded corners=5pt] (\twidth-1.9\unitlength,5+\lowerhalfbar) %b
                                      -- (\twidth-1.9\unitlength,.5) % m
                                      -- (\twidth-1.4\unitlength,.5) % m'
                                      -- (\twidth-1.4\unitlength,5+\lowerhalfbar); %e

\draw[decorate,decoration=zigzag] {(\twidth-2.4\unitlength, 1) -- (\twidth-1.9\unitlength,1)};
\draw[decorate,decoration=zigzag] {(\twidth-2.4\unitlength, 3) -- (\twidth-1.9\unitlength,3)};

\end{tikzpicture}

