\Section{Transacties}

In While is de atomaire eenheid van executie de enkelvoudige statement.

Wij breiden While uit met de notie van transacties. \Index[transactie]{Transacties} zijn arbitraire verzamelingen van
statements die gebundeld worden tot atomaire eenheden. Het speciale van deze verzamelingen is dat ze de eigenschap delen
dat taken\footnote{Taken kunnen net als het concept van parallellisme individueel voorkomen, dus waar hierboven ``taken''
genoemd is, kan ook sprake zijn van \'e\'en taak.} die met elkaar verbonden zijn ofwel allemaal slagen ofwel allemaal
niet slagen. Met slagen wordt hier bedoeld dat het programma naar een volgende state gaat, en met niet slagen
bedoelen we dat de state van voor de transactie hersteld wordt. In beide gevallen zal het programma verdergaan met de
statementsequentie. Dat wil zeggen dat er een afhankelijkheid is tussen de taken die gebaseerd is op het kunnen
verkrijgen van resultaten.

\begin{tikzpicture}
\setcounter{threadnum}{1}

% Instance 3
\draw[dotted] (0,      \thethreadnum*2-1)
		-- (\twidth,\thethreadnum*2-1)
		node [midway, right=.8\unitlength, below=0.75em] {};
\path (0,0)+(0,1) node[inststyle] (inst1) {\(T_3\)};
\draw[fill=gray!30] (4,\thethreadnum*2-1+\upperhalfbar)
				-- (4,\thethreadnum*2-1+\lowerhalfbar)
				-- (\twidth-2.4\unitlength,\thethreadnum*2-1+\lowerhalfbar)
				--	(\twidth-2.4\unitlength,\thethreadnum*2-1+\upperhalfbar)-- cycle;

\stepcounter{threadnum}

% Instance 2
\draw[dotted] (0,      \thethreadnum*2-1)
		-- (\twidth,\thethreadnum*2-1)
		node [midway, right=.8\unitlength, below=0.75em] {};
\path (0,1)+(0,2) node[inststyle] (inst3) {\(T_2\)};
\draw[fill=gray!30] (4,\thethreadnum*2-1+\upperhalfbar)
				-- (4,\thethreadnum*2-1+\lowerhalfbar)
				-- (\twidth-2.4\unitlength,\thethreadnum*2-1+\lowerhalfbar)
				--	(\twidth-2.4\unitlength,\thethreadnum*2-1+\upperhalfbar)-- cycle;

\stepcounter{threadnum}

% Thread 1
\draw[dotted] (0,      \thethreadnum*2-1)
		-- (\twidth,\thethreadnum*2-1)
		node [midway, right=.8\unitlength, above=0.75em] {};
\path (0,2)+(0,3) node[inststyle] (inst2) {\(T_1\)};
\draw[fill=gray!30] (2,\thethreadnum*2-1+\upperhalfbar)
				-- (2,\thethreadnum*2-1+\lowerhalfbar)
				-- (\twidth-.4\unitlength,\thethreadnum*2-1+\lowerhalfbar)
				-- (\twidth-.4\unitlength,\thethreadnum*2-1+\upperhalfbar)-- cycle;

\stepcounter{threadnum}

% Result/argument arrows
%t3
\node (b) at (3.5,5) {};
\node (m) at (3.5,1) {};
\node (e) at (4.1,1) {};
\draw[->,>=angle 60] (b) |- (m) -- (e) node {};

%t2
\node (b) at (3.5,5) {};
\node (m) at (3.5,3) {};
\node (e) at (4.1,3) {};
\draw[->,>=angle 60] (b) |- (m) -- (e) node {};

\draw[->,>=angle 60,rounded corners=5pt] (\twidth-1.9\unitlength,5+\lowerhalfbar) %b
									-- (\twidth-1.9\unitlength,.5) % m
									-- (\twidth-1.4\unitlength,.5) % m'
									-- (\twidth-1.4\unitlength,5+\lowerhalfbar); %e

\draw[decorate,decoration=zigzag] {(\twidth-2.4\unitlength, 1) -- (\twidth-1.9\unitlength,1)}; %t3
\draw[decorate,decoration=zigzag] {(\twidth-2.4\unitlength, 3) -- (\twidth-1.9\unitlength,3)}; %t2

\end{tikzpicture}

Een schematische weergave van het concept parallellisme. Er is een hoofdtaak \(T_1\) die ervoor zorgt dat \(T_2\) en
\(T_3\) ontstaan zodat die zelfstandig berekeningen kunnen uitvoeren. Als de taken klaar zijn is er de mogelijkheid om
de resultaten te verenigigen.

Transacties voldoen normaliter aan ACID. We leggen de nadruk op ACI, daar waar de durability vooral in praktische
situaties erg van belang is zoals beschreven in \cite{TransactionConcept}.
\begin{description}
\item[Atomair (Atomic)] Dit betekent dat als er een onderdeel van de transactie misgaat, dat de hele transactie
misgaat. Dit houdt in dat om een transactie te kunnen laten slagen, dat alle taken die betrekking hebben op de
transactie moeten slagen, en wanneer er een taak niet slaagt, alle taken niet slagen.

Het atomaire slaat dus op de transactie als geheel, aangezien die als eenheid moet worden behandeld.
\item[Consistent (Consistency)] Consistentie is iets wat afgehandeld kan worden binnen de functies. Omdat While een
getypeerde taal is, hebben we sowieso al niet te maken met type-fouten. Wel kunnen we domeinfouten ontstaan,
bijvoorbeeld het banksaldo mag niet onder \(0\) komen; dit moet door de auteur van de functie afgehandeld worden, de
taal While, en zelfs het programma als geheel, kennen beiden geen domeinbeperkingen.

Gedurende de berekeningen mogen de beperkingen gelaten worden voor wat ze zijn, maar alvorens het aanroepen van
\kw{set_result} is het de taak van de functie om gecontroleerd te hebben of het resultaat \"uberhaupt geldig is.
\item[Ge\"isoleerd (Isolated)] Deze eigenschap hebben we erg duidelijk opgenomen in onze specificaties door ervoor te
zorgen dat variabelen en states nooit ``by reference'' worden meegegeven. Dit zorgt er in essentie voor dat elke taak
zijn eigen set met variabelen heeft en dat die veranderd kunnen worden zonder dat dit ook maar enig effect heeft op de
variabelen van andere taken.

Het is natuurlijk wel nodig om resultaten bij elkaar te kunnen laten komen, omdat transacties anders beperkt nut
hebben. Dit doen wij door gebruik te maken van de executieomgeving die wederom zonder zijeffecten netjes in sommige
gevallen variabelen in taken aan kan passen zodat bijvoorbeeld resultaten van de door de caller gespawnde taken in de
caller als resultaat geplaatst kunnen worden.
\item[Duurzaam (Durability)] Wij houden geen logs bij van welke wijzigingen we wanneer uitgevoerd hebben om welke
reden. Dit zou je doorgaans wel willen doen wanneer je waarde hecht aan de duurzaamheid.

Om die reden kunnen we ook niet, wanneer de main-taak een abort krijgt, de taak opnieuw opstarten in dezelfde toestand
als waar hij gebleven was; of opstarten in een geldige toestand, en dan vanuit het log opnieuw dezelfde keuzes maken om
weer in dezelfde toestand te komen.
\end{description}

Om transactionele omgevingen te kunnen voorzien van een naam, breiden we de meta-variabelen en syntactische
categorie\"en uit:
\begin{equation*}
\begin{array}{rl}
  	tid & \textrm{geeft een \Index{transactielabel} aan, \syncat{TransactionLabel}} \\
\end{array}
\end{equation*}
\newpage
\Subsection{Transactie}

Alle statements binnen een taak en alle taken die gespawnd worden binnen deze taak
vormen deel van deze transactie. De transactie wordt gestart met \kw{start_transaction}. De transactie wordt afgesloten
met een \kw{commit_transaction} of een \kw{rollback_transaction}. De taak die een transactie start wordt de
transactiemanager van de transactie genoemd.

De transactie-omgeving is de verzameling van states en statements van de taken die onderdeel uitmaken van de transactie. Deze
omgeving is bekend onder zijn transactielabel, \(tid\). De transactielabels zijn globaal bekend.

\begin{functie}{start_transaction $tid$} \\
\strut\qquad $S'$ \\
\kw{if} \functienaam{collect_votes} $tid$ \\
\kw{then} \\
\strut\qquad \kw{if} $b$ \\
\strut\qquad \kw{then} \\
\strut\qquad\qquad $S''$ \\
\strut\qquad\qquad\functienaam{commit_transaction} $tid$ \\
\strut\qquad \kw{else} \\
\strut\qquad\qquad $S'''$ \\
\strut\qquad\qquad\functienaam{rollback_transaction} $tid$ \\
\strut\kw{else} \\
\strut\qquad $S''''$ \\
\strut\qquad\functienaam{rollback_transaction} $tid$
\functieomschrijving Omwille van de
overzichtelijkheid wordt deze functie opgedeeld in vier stukken.
\end{functie}

\Subsubsection[Starttransactie]{start_transaction}
\kw{start_transaction} markeert het begin van een transactie.

De enige voorwaarde om een transactie op te kunnen starten is dat het een uniek transactielabel krijgt en het label mag
niet \kw{undef} zijn.

Bij \kw{start_transaction} wordt een kopie gemaakt van de huidige state, en deze kopie wordt opgeslagen in de state. En
het transactielabel wordt toegevoegd aan de set met transactielabels.

\begin{functie}{start_transaction $tid$}
	\functieargument{$tid$: Transactielabel}
\end{functie}

\Subsubsection[Collectvotes]{collect_votes}
Verzamelt van alle uitstaande subtaken binnen de transactie het resultaat en als alle resultaten
\kw{true} zijn, geeft \kw{collect_votes} \kw{true} weer, anders \kw{false}.

Als \kw{collect_votes} wordt aangeroepen voor taken die nog geen \kw{set_result} hebben aangeroepen dan krijgt
\kw{collect_votes} de waarde \kw{false}.

\begin{functie}{collect_votes $tid$}
	\functieargument{$tid$: Transactielabel}
	\functieresultaat{} Levert een \kw{true} op in het geval dat alle statusTaken van alle
	taken die binnen deze transactie (\(tid\)) opgestart zijn, \kw{true} zijn, \kw{false} anders
\end{functie}

\Subsubsection[Committransactie]{commit_transaction}
Alle wijzigingen die binnen de transactie in de state van het programma gemaakt zijn
worden nu definitief gemaakt en worden behouden. De transactielabel wordt verwijderd en daarmee verdwijnt de
transactieomgeving.

\kw{commit_transaction} deblokkeert alle \functienaam{set_results} van alle onderliggende deelnemende taken met
\kw{true}. De functie blokkeert totdat alle eventuele subtaken die onderdeel van de transactie zijn
\kw{end_commit_transaction} hebben aangroepen ten teken van het feit dat zijn alle statements die noodzakelijk zijn
om de commit van de transactie voor die taak definitief te maken zijn uitgevoerd.

\begin{functie}{commit_transaction $tid$}
	\functieargument{$tid$: Transactielabel}
\end{functie}

\Subsubsection[Rollbacktransactie]{rollback_transaction}
De state van een taak gaat na een rollback terug naar de opgeslagen state. De statementsequentie gaat niet terug naar
dat punt, maar gaat verder. De transactielabel wordt verwijderd en daarmee verdwijnt de transactieomgeving.

Deblokkeert alle \functienaam{set_result}s van alle onderliggende deelnemende taken met \kw{false}. Indien dit het geval is
wordt de huidige state gereset naar een eerder opgeslagen state zoals opgeslagen is door de huidige taak met
\functienaam{start_transactie}. De functie blokkeert totdat alle eventuele subtaken die onderdeel van de transactie
zijn \kw{end_rollback_transaction} hebben aangroepen ten teken van het feit dat zijn alle statements die noodzakelijk zijn
om de rollback van de transactie voor die taak definitief te maken zijn uitgevoerd.

\begin{functie}{rollback_transaction $tid$}
	\functieargument{$tid$: Transactielabel}
\end{functie}

\Subsection{Einde van een transactie door een taak}

Als een transactiemanager een \kw{commit_transaction} of \kw{rollback_transaction} uitvoert dan wordt de
\kw{set_result} van de onderliggende taken gedeblokkeerd, waarna deze de statements behorend bij \kw{true} of \kw{false} uitvoeren.
Zodra de onderliggende taak daarmee gereed is maakt deze dit via de executieomgeving met behulp van
\kw{end_commit_transaction} (in het geval van \kw{true}) of \kw{end_commit_transaction} (in het geval van \kw{false})
bekend.

Zodra alle onderliggende taken het einde van de transactie bekend hebben gesteld zal de \kw{commit_transaction} of
\kw{rollback_transaction} gedeblokkeerd worden.

\begin{functie}{end_commit/rollback_transaction} \\
\kw{if}\:\kw{set_result} $tl$ $vn$ $a$ $b$ $tid$ \kw{then} \\
\strut\qquad S_1\:\kw{end_commit_transaction} $tl$ $tid$  \\
\strut\kw{else} \\
\strut\qquad S_2\:\kw{end_rollback_transaction} $tl$ $tid$
\functieomschrijving De taak die de transactie gestart is wordt via de executeomgeving op de hoogte gesteld van het feit
dat deze taak alle statements met betrekking tot een commit of rollback heeft uitgevoerd

Omwille van de overzichtelijkheid wordt deze statement opgedeeld in twee stukken
\end{functie}

\Subsubsection[Committransactie]{end_commit_transaction}

Met behulp van \kw{end_commit_transaction} maakt een taak bekend dat alle statements met betrekking tot een commit zijn
uitgevoerd.

\begin{functie}{end_commit_transaction $tl$ $tid$}
	\functieargument{$tl$: Taaklabel}
	\functieargument{$tid$: Transactielabel}
\end{functie}

\Subsubsection[Rollbacktransactie]{end_rollback_transaction}

Met behulp van \kw{end_rollback_transaction} maakt een taak bekend dat alle statements met betrekking tot een rollback
zijn uitgevoerd.

\begin{functie}{end_rollback_transaction $tl$ $tid$}
	\functieargument{$tl$: Taaklabel}
	\functieargument{$tid$: Transactielabel}
\end{functie}

\Subsection[Individueletransactieafhandeling]{Individuele transactieafhandeling}
De taken zijn verantwoordelijk voor het uitvoeren van een rollback wanneer daar door de caller om gevraagd wordt.

Hoewel in kleine While-programma zonder zijeffecten het nut niet direct zichtbaar is, is er toch voor gekozen om de
individuele taken verantwoordelijk te stellen voor het verzorgen van hun eigen commit of rollback.

In een echt programma is de caller doorgaans niet op de hoogte van de implementatie van de callee, en daardoor kan de
caller niets zeggen over de acties die ondernomen moeten worden door de callee bij een commit of een rollback.
