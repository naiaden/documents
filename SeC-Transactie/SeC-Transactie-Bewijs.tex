\Section{Bewijs} \nocite{SkeenStonebraker}
De bewijzen zullen vanwege de omvang gepresenteerd worden in twee vormen. Allereerst zal in dit hoofdstuk de korte
versie vermeld staan. Deze korte versie laat niet steeds alle statements zien, maar toont wel de functienaam van de
functie die gebruikt wordt, en toont de huidige state op dat moment. Met behulp van deze versie is het mogelijk het
grote bewijs na te gaan, of zelf het bewijs te volgen wanneer de statements allemaal uitgeschreven worden.

De tweede versie is de complete versie, dus compleet met functienamen, states en de complete statementsequentie. Omdat
de grootte van deze versies weinig gering zijn, zijn ze als los bestand meegeleverd. Dit bestand is te bezichtigen op
\url{http://trac.askesis.nl/tidbits/browser/SeC/aux/Sec-Transactie-Bewijs.ods} of op
\url{http://trac.askesis.nl/tidbits/browser/SeC/aux/Sec-Transactie-Bewijs.xls} of via de bijlage bij dit document.

Elke bewijsvoering wordt begeleid door een afbeelding die een schets van de opbouw van het programma weergeeft. In deze
afbeeldingen zijn markers en in het bewijs zullen deze markers ook verschijnen zodat de lezer zich beter kan ori\"enteren
en het duidelijker is in welke context een bepaalde state zich bevindt.

Het bewijs dient van boven naar beneden gelezen te worden, behalve de taken op zichzelf, die mogen parallel bekeken
worden (we kennen toch geen racecondities). De state aangegeven is de state die geldt op het moment dat je de functie
besluit te kiezen. Het resultaat van die functie staat dus \'e\'en regel lager. Voor een complete bewijsboom, zie
bijlage.

\Subsection{Commit}
Bij een commit willen we graag bewijzen dat beide taken een antwoord teruggeven aan de hoofdtaak, en dat deze
antwoorden verkregen zijn op een manier waarvan de taak dacht dat het goed was.

In dit bewijs worden twee taken gespawned uit \(main\) die de taken hebben om \(1\) bij \(2\) op te tellen, en om \(3\)
bij \(4\) op te tellen. We verwachten dus dat \(main\) respectievelijk de waarden \(3\) en \(7\) binnenkrijgt. Als
\(main\) deze waarden vervolgens optelt moet aan het einde van het programma \kw{grand_total} de waarde 10 hebben.

\begin{tikzpicture}
\setcounter{threadnum}{1}

\draw[->,>=angle 60,rounded corners=5pt] (\twidth-2.65\unitlength,5+\lowerhalfbar) %b
									-- (\twidth-2.65\unitlength,.5) % m
									-- (\twidth-2.15\unitlength,.5) % m'
									-- (\twidth-2.15\unitlength,5+\lowerhalfbar); %e

\draw[->,>=angle 60,rounded corners=5pt] (\twidth-1.65\unitlength,5+\lowerhalfbar) %b
									-- (\twidth-1.65\unitlength,.5) % m
									-- (\twidth-1.15\unitlength,.5) % m'
									-- (\twidth-1.15\unitlength,5+\lowerhalfbar); %e

\draw[] (2.5,\thethreadnum*4-1+2.75) % transactional env
				-- (2.5,\thethreadnum*2-2)
				-- (\twidth-.9\unitlength,\thethreadnum*2-2)
				--	(\twidth-.9\unitlength,\thethreadnum*4-1+2.75)-- cycle;
\node at (\twidth-.9\unitlength,\thethreadnum*4-1+2.75) [below left]{\textbf{transactie1}};

% Instance 3
\draw[dotted] (0,      \thethreadnum*2-1)
		-- (\twidth,\thethreadnum*2-1)
		node [midway, right=.8\unitlength, below=0.75em] {};
\path (0,0)+(0,1) node[inststyle] (inst1) {taak2};
\draw[fill=gray!30] (3.15,\thethreadnum*2-1+\upperhalfbar)
				-- (3.15,\thethreadnum*2-1+\lowerhalfbar)
				-- (\twidth-3.1\unitlength,\thethreadnum*2-1+\lowerhalfbar)
				--	(\twidth-3.1\unitlength,\thethreadnum*2-1+\upperhalfbar)-- cycle;
\draw[fill=gray!30] (\twidth-2.15\unitlength,\thethreadnum*2-1+\upperhalfbar)
				-- (\twidth-2.15\unitlength,\thethreadnum*2-1+\lowerhalfbar)
				-- (\twidth-1.65\unitlength,\thethreadnum*2-1+\lowerhalfbar)
				--	(\twidth-1.65\unitlength,\thethreadnum*2-1+\upperhalfbar)-- cycle;

\stepcounter{threadnum}

% Instance 2
\draw[dotted] (0,      \thethreadnum*2-1)
		-- (\twidth,\thethreadnum*2-1)
		node [midway, right=.8\unitlength, below=0.75em] {};
\path (0,1)+(0,2) node[inststyle] (inst3) {taak1};
\draw[fill=gray!30] (3.15,\thethreadnum*2-1+\upperhalfbar)
				-- (3.15,\thethreadnum*2-1+\lowerhalfbar)
				-- (\twidth-3.4\unitlength,\thethreadnum*2-1+\lowerhalfbar)
				--	(\twidth-3.4\unitlength,\thethreadnum*2-1+\upperhalfbar)-- cycle;
\draw[fill=gray!30] (\twidth-2.15\unitlength,\thethreadnum*2-1+\upperhalfbar)
				-- (\twidth-2.15\unitlength,\thethreadnum*2-1+\lowerhalfbar)
				-- (\twidth-1.65\unitlength,\thethreadnum*2-1+\lowerhalfbar)
				--	(\twidth-1.65\unitlength,\thethreadnum*2-1+\upperhalfbar)-- cycle;


\stepcounter{threadnum}

% Thread 1
\draw[dotted] (0,      \thethreadnum*2-1)
		-- (\twidth,\thethreadnum*2-1)
		node [midway, right=.8\unitlength, above=0.75em] {};
\path (0,2)+(0,3) node[inststyle] (inst2) {main};
\draw[fill=gray!30] (2,\thethreadnum*2-1+\upperhalfbar)
				-- (2,\thethreadnum*2-1+\lowerhalfbar)
				-- (\twidth-.4\unitlength,\thethreadnum*2-1+\lowerhalfbar)
				-- (\twidth-.4\unitlength,\thethreadnum*2-1+\upperhalfbar)-- cycle;

\draw[dotted] (2,5)--(2,6.25) node[marker] () {\footnotesize a};
\draw[dotted] (2.5,5)--(2.5,6.25) node[marker] () {\footnotesize b};
\draw[dotted] (2.75,5)--(2.75,6.25) node[marker] () {\footnotesize c};
\draw[dotted] (3,5)--(3,6.25) node[marker] () {\footnotesize d};
\draw[dotted] (\twidth-3.4\unitlength,3)--(\twidth-3.4\unitlength,6.25) node[marker] () {\footnotesize t};
\draw[dotted] (\twidth-3.1\unitlength,1)--(\twidth-3.1\unitlength,6.25) node[marker] () {\footnotesize u};
\draw[dotted] (\twidth-2.65\unitlength,5)--(\twidth-2.65\unitlength,6.25) node[marker] () {\footnotesize v};
\draw[dotted] (\twidth-2.15\unitlength,5)--(\twidth-2.15\unitlength,6.25) node[marker] () {\footnotesize w};
%\draw[dotted] (\twidth-1.65\unitlength,1)--(\twidth-1.65\unitlength,6.25) node[marker] () {\footnotesize x};
%\draw[dotted] (\twidth-1.15\unitlength,3)--(\twidth-1.15\unitlength,6.25) node[marker] () {\footnotesize y};
\draw[dotted] (\twidth-.9\unitlength,5)--(\twidth-.9\unitlength,6.25) node[marker] () {\footnotesize z};


% Result/argument arrows
%t3
\node (b) at (2.75,5) {};
\node (m) at (2.75,1) {};
\node (e) at (3.25,1) {};
\draw[->,>=angle 60] (b) |- (m) -- (e) node {};

%t2
\node (b) at (3,5) {};
\node (m) at (3,3) {};
\node (e) at (3.25,3) {};
\draw[->,>=angle 60] (b) |- (m) -- (e) node {};



\draw[decorate,decoration=zigzag] {(\twidth-3.1\unitlength, 1) -- (\twidth-2.65\unitlength,1)}; %t3
\draw[decorate,decoration=zigzag] {(\twidth-3.4\unitlength, 3) -- (\twidth-2.65\unitlength,3)}; %t2

\end{tikzpicture}

Dit is het programma zoals we gebruiken voor het bewijs:

\begin{lstlisting}[caption={Commit}]
func optellen tl tl vn tid var x:=2; var y:=3; epsilon is
start
	result := x + y;
	if result < 10
	then
		valid := true
	else
		valid := false
	if set_result tl vn result valid tid then
		skip end_commit_transaction tl tid
	else
		skip end_rollback_transaction tl tid
end;
func main tl tl vn tid epsilon is
start
	start_transaction transactie1
	main_1:=undef;
	main_2:=undef;
	spawn fl:=optellen tl_callee:=taak1 tl_caller:=main vn:=main_1 tid:=transactie1
		var x:=1; var y:=2; epsilon;
	spawn fl:=optellen tl_callee:=taak2 tl_caller:=main vn:=main_2 tid:=transactie1
		var x:=3; var y:=4; epsilon;
	if wait taak1 then skip else skip;
	if wait taak2 then skip else skip
	if collect_votes transactie1 then
		grand_total := main_1 + main_2;
		if grand_total < 15 then
			skip;
			commit_transaction transactie1
		else
			skip;
			rollback_transaction transactie1
	else
		grand_total := 0;
		rollback_transaction transactie1
end;
epsilon
\end{lstlisting}

\(
\begin{array}{rl}
& Executieomgeving \\\hline
\kw{func1} & s  \\
\kw{func1} & s  \\
\kw{func2} & s  \\
\kw{spawn} & s  \\
& s \\
& \\
& main\marker{a} \\\hline
\kw{var2} & s[\kw{tl}\mapsto\kw{\kw{\kw{undef}}}][\kw{tid}\mapsto \kw{\kw{undef}}][\kw{vn}\mapsto \kw{\kw{undef}}]  \\
\kw{var2} & s[\kw{tl}\mapsto\kw{\kw{\kw{undef}}}][\kw{tid}\mapsto \kw{\kw{undef}}][\kw{vn}\mapsto \kw{\kw{undef}}]  \\
\kw{call} & s[\kw{tl}\mapsto \kw{\kw{undef}}][\kw{tid}\mapsto \kw{\kw{undef}}][\kw{vn}\mapsto \kw{\kw{undef}}]\marker{b}  \\
\kw{transaction} & s[\kw{tl}\mapsto \kw{\kw{undef}}][\kw{tid}\mapsto \kw{\kw{undef}}][\kw{vn}\mapsto \kw{\kw{undef}}] \\
\kw{ass} & s[\kw{tl}\mapsto \kw{\kw{undef}}][\kw{tid}\mapsto transactie1][\kw{vn}\mapsto \kw{\kw{undef}}] \\
\kw{ass} & s[\kw{tl}\mapsto \kw{\kw{undef}}][\kw{tid}\mapsto transactie1][\kw{vn}\mapsto \kw{\kw{undef}}][\kw{main_1}\mapsto\kw{\kw{undef}}]\marker{c} \\
\kw{spawn} & s[\kw{tl}\mapsto \kw{\kw{undef}}][\kw{tid}\mapsto transactie1][\kw{vn}\mapsto \kw{\kw{undef}}][\kw{main_1}\mapsto\kw{\kw{undef}}][\kw{main_2}\mapsto \kw{\kw{undef}}]\marker{d} \\
\kw{spawn} & s[\kw{tl}\mapsto \kw{\kw{undef}}][\kw{tid}\mapsto transactie1][\kw{vn}\mapsto \kw{\kw{undef}}][\kw{main_1}\mapsto \kw{\kw{undef}}][\kw{main_2}\mapsto \kw{\kw{undef}}] \\
\kw{if tt} & s[\kw{tl}\mapsto \kw{\kw{undef}}][\kw{tid}\mapsto transactie1][\kw{vn}\mapsto \kw{\kw{undef}}][\kw{main_1}\mapsto \kw{\kw{undef}}][\kw{main_2}\mapsto \kw{\kw{undef}}] \\

\kw{skip} & s[\kw{tl}\mapsto \kw{\kw{undef}}][\kw{tid}\mapsto transactie1][\kw{vn}\mapsto \kw{\kw{undef}}][\kw{main_1}\mapsto 3][\kw{main_2}\mapsto \kw{\kw{undef}}] \\
\kw{if tt} & s[\kw{tl}\mapsto \kw{\kw{undef}}][\kw{tid}\mapsto transactie1][\kw{vn}\mapsto \kw{\kw{undef}}][\kw{main_1}\mapsto 3][\kw{main_2}\mapsto \kw{\kw{undef}}] \\
\kw{skip} & s[\kw{tl}\mapsto \kw{\kw{undef}}][\kw{tid}\mapsto transactie1][\kw{vn}\mapsto \kw{\kw{undef}}][\kw{main_1}\mapsto3][\kw{main_2}\mapsto 7]\marker{v} \\
\kw{if tt} & s[\kw{tl}\mapsto \kw{\kw{undef}}][\kw{tid}\mapsto transactie1][\kw{vn}\mapsto \kw{\kw{undef}}][\kw{main_1}\mapsto3][\kw{main_2}\mapsto 7]\marker{w} \\
\kw{ass} & s[\kw{tl}\mapsto \kw{\kw{undef}}][\kw{tid}\mapsto transactie1][\kw{vn}\mapsto \kw{\kw{undef}}][\kw{main_1}\mapsto 3][\kw{main_2}\mapsto 7] \\
\kw{if tt} & s[\kw{tl}\mapsto \kw{\kw{undef}}][\kw{tid}\mapsto transactie1][\kw{vn}\mapsto \kw{\kw{undef}}][\kw{main_1}\mapsto 3][\kw{main_2}\mapsto 7] \\
          & \hspace{\sLength}[\kw{grand_total}\mapsto 10] \\
\kw{skip} & s[\kw{tl}\mapsto \kw{\kw{undef}}][\kw{tid}\mapsto transactie1][\kw{vn}\mapsto \kw{\kw{undef}}][\kw{main_1}\mapsto3][\kw{main_2}\mapsto 7] \\
          & \hspace{\sLength}[\kw{grand_total}\mapsto 10]\marker{z} \\
\kw{commit} & s[\kw{tl}\mapsto \kw{\kw{undef}}][\kw{tid}\mapsto transactie1][\kw{vn}\mapsto \kw{\kw{undef}}][\kw{main_1}\mapsto 3][\kw{main_2}\mapsto 7] \\
          & \hspace{\sLength}[\kw{grand_total}\mapsto 10] \\
& s[\kw{tl}\mapsto \kw{\kw{undef}}][\kw{tid}\mapsto transactie1][\kw{vn}\mapsto \kw{\kw{undef}}][\kw{main_1}\mapsto 3][\kw{main_2}\mapsto 7] \\
          & \hspace{\sLength}[\kw{grand_total}\mapsto 10] \\
& \\
& taak1 \\\hline
\kw{var1} & s[\kw{tl}\mapsto \kw{\kw{undef}}][\kw{tid}\mapsto transactie1][\kw{vn}\mapsto main_1][\kw{main_1}\mapsto \kw{\kw{undef}}][\kw{main_2}\mapsto \kw{\kw{undef}}] \\
          & \hspace{\sLength}[\kw{tl_caller}\mapsto main][\kw{tl_callee}\mapsto taak1]  \\
\kw{var1} & s[\kw{tl}\mapsto \kw{\kw{undef}}][\kw{tid}\mapsto transactie1][\kw{vn}\mapsto main_1][\kw{main_1}\mapsto \kw{\kw{undef}}][\kw{main_2}\mapsto \kw{\kw{undef}}] \\
          & \hspace{\sLength}[\kw{tl_caller}\mapsto main][\kw{tl_callee}\mapsto taak1][\kw{x}\mapsto 2]  \\
\kw{var2} & s[\kw{tl}\mapsto \kw{\kw{undef}}][\kw{tid}\mapsto transactie1][\kw{vn}\mapsto main_1][\kw{main_1}\mapsto \kw{\kw{undef}}][\kw{main_2}\mapsto \kw{\kw{undef}}] \\
          & \hspace{\sLength}[\kw{tl_caller}\mapsto main][\kw{tl_callee}\mapsto taak1][\kw{x}\mapsto 2][\kw{y}\mapsto 3]  \\
\kw{var1} & s[\kw{tl}\mapsto \kw{\kw{undef}}][\kw{tid}\mapsto transactie1][\kw{vn}\mapsto main_1][\kw{main_1}\mapsto \kw{\kw{undef}}][\kw{main_2}\mapsto \kw{\kw{undef}}] \\
          & \hspace{\sLength}[\kw{tl_caller}\mapsto main][\kw{tl_callee}\mapsto taak1][\kw{x}\mapsto 2][\kw{y}\mapsto 3]  \\
\kw{var1} & s[\kw{tl}\mapsto \kw{\kw{undef}}][\kw{tid}\mapsto transactie1][\kw{vn}\mapsto main_1][\kw{main_1}\mapsto \kw{\kw{undef}}][\kw{main_2}\mapsto \kw{\kw{undef}}] \\
          & \hspace{\sLength}[\kw{tl_caller}\mapsto main][\kw{tl_callee}\mapsto taak1][\kw{x}\mapsto 1][\kw{y}\mapsto 3]  \\
\kw{var2} & s[\kw{tl}\mapsto \kw{\kw{undef}}][\kw{tid}\mapsto transactie1][\kw{vn}\mapsto main_1][\kw{main_1}\mapsto \kw{\kw{undef}}][\kw{main_2}\mapsto \kw{\kw{undef}}] \\
          & \hspace{\sLength}[\kw{tl_caller}\mapsto main][\kw{tl_callee}\mapsto taak1][\kw{x}\mapsto 1][\kw{y}\mapsto 2]  \\
\kw{call} & s[\kw{tl}\mapsto \kw{\kw{undef}}][\kw{tid}\mapsto transactie1][\kw{vn}\mapsto main_1][\kw{main_1}\mapsto \kw{\kw{undef}}][\kw{main_2}\mapsto \kw{\kw{undef}}] \\
          & \hspace{\sLength}[\kw{tl_caller}\mapsto main][\kw{tl_callee}\mapsto taak1][\kw{x}\mapsto 1][\kw{y}\mapsto 2]  \\
\kw{ass} & s[\kw{tl}\mapsto \kw{\kw{undef}}][\kw{tid}\mapsto transactie1][\kw{vn}\mapsto main_1][\kw{main_1}\mapsto \kw{\kw{undef}}][\kw{main_2}\mapsto \kw{\kw{undef}}] \\
          & \hspace{\sLength}[\kw{tl_caller}\mapsto main][\kw{tl_callee}\mapsto taak1][\kw{x}\mapsto 1][\kw{y}\mapsto 2]  \\
\kw{if tt} & s[\kw{tl}\mapsto \kw{\kw{undef}}][\kw{tid}\mapsto transactie1][\kw{vn}\mapsto main_1][\kw{main_1}\mapsto \kw{\kw{undef}}][\kw{main_2}\mapsto \kw{\kw{undef}}] \\
          & \hspace{\sLength}[\kw{tl_caller}\mapsto main][\kw{tl_callee}\mapsto taak1][\kw{x}\mapsto 1][\kw{y}\mapsto 2][\kw{result}\mapsto 3]  \\
\kw{ass} & s[\kw{tl}\mapsto \kw{\kw{undef}}][\kw{tid}\mapsto transactie1][\kw{vn}\mapsto main_1][\kw{main_1}\mapsto \kw{\kw{undef}}][\kw{main_2}\mapsto \kw{\kw{undef}}] \\
          & \hspace{\sLength}[\kw{tl_caller}\mapsto main][\kw{tl_callee}\mapsto taak1][\kw{x}\mapsto1][\kw{y}\mapsto2][\kw{result}\mapsto 3]\marker{t}  \\
\kw{if tt} & s[\kw{tl}\mapsto \kw{\kw{undef}}][\kw{tid}\mapsto transactie1][\kw{vn}\mapsto main_1][\kw{main_1}\mapsto \kw{\kw{undef}}][\kw{main_2}\mapsto \kw{\kw{undef}}] \\
          & \hspace{\sLength}[\kw{tl_caller}\mapsto main][\kw{tl_callee}\mapsto taak1][\kw{x}\mapsto 1][\kw{y}\mapsto 2][\kw{result}\mapsto 3][\kw{valid}\mapsto \kw{true}]  \\
\end{array}
\)

\(
\begin{array}{rl}
\kw{end trans tt} & s[\kw{tl}\mapsto \kw{\kw{undef}}][\kw{tid}\mapsto transactie1][\kw{vn}\mapsto main_1][\kw{main_1}\mapsto \kw{\kw{undef}}][\kw{main_2}\mapsto \kw{\kw{undef}}] \\
          & \hspace{\sLength}[\kw{tl_caller}\mapsto main][\kw{tl_callee}\mapsto taak1][\kw{x}\mapsto 1][\kw{y}\mapsto 2][\kw{result}\mapsto 3][\kw{valid}\mapsto \kw{true}]  \\
\kw{skip} & s[\kw{tl}\mapsto \kw{\kw{undef}}][\kw{tid}\mapsto transactie1][\kw{vn}\mapsto main_1][\kw{main_1}\mapsto\kw{\kw{undef}}][\kw{main_2}\mapsto \kw{\kw{undef}}] \\
          & \hspace{\sLength}[\kw{tl_caller}\mapsto main][\kw{tl_callee}\mapsto taak1][\kw{x}\mapsto 1][\kw{y}\mapsto2][\kw{result}\mapsto 3][\kw{valid}\mapsto \kw{true}]  \\
&s[\kw{tl}\mapsto \kw{\kw{undef}}][\kw{tid}\mapsto transactie1][\kw{vn}\mapsto main_1][\kw{main_1}\mapsto\kw{\kw{undef}}][\kw{main_2}\mapsto \kw{\kw{undef}}] \\
          & \hspace{\sLength}[\kw{tl_caller}\mapsto main][\kw{tl_callee}\mapsto taak1][\kw{x}\mapsto 1][\kw{y}\mapsto2][\kw{result}\mapsto 3][\kw{valid}\mapsto \kw{true}] \\
& \\
& taak2 \\\hline
\kw{var1} & s[\kw{tl}\mapsto \kw{\kw{undef}}][\kw{tid}\mapsto transactie1][\kw{vn}\mapsto main_2][\kw{main_1}\mapsto \kw{\kw{undef}}][\kw{main_2}\mapsto \kw{\kw{undef}}] \\
          & \hspace{\sLength}[\kw{tl_caller}\mapsto main][\kw{tl_callee}\mapsto taak2]  \\
\kw{var1} & s[\kw{tl}\mapsto \kw{\kw{undef}}][\kw{tid}\mapsto transactie1][\kw{vn}\mapsto main_2][\kw{main_1}\mapsto \kw{\kw{undef}}][\kw{main_2}\mapsto \kw{\kw{undef}}] \\
          & \hspace{\sLength}[\kw{tl_caller}\mapsto main][\kw{tl_callee}\mapsto taak2][\kw{x}\mapsto 2]  \\
\kw{var2} & s[\kw{tl}\mapsto \kw{\kw{undef}}][\kw{tid}\mapsto transactie1][\kw{vn}\mapsto main_2][\kw{main_1}\mapsto \kw{\kw{undef}}][\kw{main_2}\mapsto \kw{\kw{undef}}] \\
          & \hspace{\sLength}[\kw{tl_caller}\mapsto main][\kw{tl_callee}\mapsto taak2[\kw{x}\mapsto 2][\kw{y}\mapsto 3]  \\
\kw{var1} & s[\kw{tl}\mapsto \kw{\kw{undef}}][\kw{tid}\mapsto transactie1][\kw{vn}\mapsto main_2][\kw{main_1}\mapsto \kw{\kw{undef}}][\kw{main_2}\mapsto \kw{\kw{undef}}] \\
          & \hspace{\sLength}[\kw{tl_caller}\mapsto main][\kw{tl_callee}\mapsto taak2][\kw{x}\mapsto 2][\kw{y}\mapsto 3]  \\
\kw{var1} & s[\kw{tl}\mapsto \kw{\kw{undef}}][\kw{tid}\mapsto transactie1][\kw{vn}\mapsto main_2][\kw{main_1}\mapsto \kw{\kw{undef}}][\kw{main_2}\mapsto \kw{\kw{undef}}] \\
          & \hspace{\sLength}[\kw{tl_caller}\mapsto main][\kw{tl_callee}\mapsto taak2][\kw{x}\mapsto 3][\kw{y}\mapsto 3]  \\
\kw{var2} & s[\kw{tl}\mapsto \kw{\kw{undef}}][\kw{tid}\mapsto transactie1][\kw{vn}\mapsto main_2][\kw{main_1}\mapsto \kw{\kw{undef}}][\kw{main_2}\mapsto \kw{\kw{undef}}] \\
          & \hspace{\sLength}[\kw{tl_caller}\mapsto main][\kw{tl_callee}\mapsto taak2][\kw{x}\mapsto 3][\kw{y}\mapsto 4]  \\
\kw{call} & s[\kw{tl}\mapsto \kw{\kw{undef}}][\kw{tid}\mapsto transactie1][\kw{vn}\mapsto main_2][\kw{main_1}\mapsto \kw{\kw{undef}}][\kw{main_2}\mapsto \kw{\kw{undef}}] \\
          & \hspace{\sLength}[\kw{tl_caller}\mapsto main][\kw{tl_callee}\mapsto taak2][\kw{x}\mapsto 3][\kw{y}\mapsto 4]  \\
\kw{ass} & s[\kw{tl}\mapsto \kw{\kw{undef}}][\kw{tid}\mapsto transactie1][\kw{vn}\mapsto main_2][\kw{main_1}\mapsto \kw{\kw{undef}}][\kw{main_2}\mapsto \kw{\kw{undef}}] \\
          & \hspace{\sLength}[\kw{tl_caller}\mapsto main][\kw{tl_callee}\mapsto taak2][\kw{x}\mapsto 3][\kw{y}\mapsto 4]  \\
\kw{if tt} & s[\kw{tl}\mapsto \kw{\kw{undef}}][\kw{tid}\mapsto transactie1][\kw{vn}\mapsto main_2][\kw{main_1}\mapsto \kw{\kw{undef}}][\kw{main_2}\mapsto \kw{\kw{undef}}] \\
          & \hspace{\sLength}[\kw{tl_caller}\mapsto main][\kw{tl_callee}\mapsto taak2][\kw{x}\mapsto 3][\kw{y}\mapsto 4][\kw{result}\mapsto 7]  \\
\kw{ass} & s[\kw{tl}\mapsto \kw{\kw{undef}}][\kw{tid}\mapsto transactie1][\kw{vn}\mapsto main_2][\kw{main_1}\mapsto \kw{\kw{undef}}][\kw{main_2}\mapsto \kw{\kw{undef}}] \\
          & \hspace{\sLength}[\kw{tl_caller}\mapsto main][\kw{tl_callee}\mapsto taak2][\kw{x}\mapsto3][\kw{y}\mapsto4][\kw{result}\mapsto 7]\markers{u}  \\
\kw{if tt} & s[\kw{tl}\mapsto \kw{\kw{undef}}][\kw{tid}\mapsto transactie1][\kw{vn}\mapsto main_2][\kw{main_1}\mapsto \kw{\kw{undef}}][\kw{main_2}\mapsto \kw{\kw{undef}}] \\
          & \hspace{\sLength}[\kw{tl_caller}\mapsto main][\kw{tl_callee}\mapsto taak2][\kw{x}\mapsto 3][\kw{y}\mapsto 4][\kw{result}\mapsto 7][\kw{valid}\mapsto \kw{true}]  \\
\kw{end trans tt} & s[\kw{tl}\mapsto\kw{\kw{undef}}][\kw{tid}\mapsto transactie1][\kw{vn}\mapsto main_2][\kw{main_1}\mapsto\kw{\kw{undef}}][\kw{main_2}\mapsto \kw{\kw{undef}}] \\
          & \hspace{\sLength}[\kw{tl_caller}\mapsto main][\kw{tl_callee}\mapsto taak2][\kw{x}\mapsto3][\kw{y}\mapsto4][\kw{result}\mapsto 7][\kw{valid}\mapsto \kw{true}]  \\
\kw{skip} & s[\kw{tl}\mapsto\kw{\kw{undef}}][\kw{tid}\mapsto transactie1][\kw{vn}\mapsto main_2][\kw{main_1}\mapsto\kw{\kw{undef}}][\kw{main_2}\mapsto \kw{\kw{undef}}] \\
          & \hspace{\sLength}[\kw{tl_caller}\mapsto main][\kw{tl_callee}\mapsto taak2][\kw{x}\mapsto3][\kw{y}\mapsto4][\kw{result}\mapsto 7][\kw{valid}\mapsto \kw{true}]  \\
          & s[\kw{tl}\mapsto\kw{\kw{undef}}][\kw{tid}\mapsto transactie1][\kw{vn}\mapsto main_2][\kw{main_1}\mapsto\kw{\kw{undef}}][\kw{main_2}\mapsto \kw{\kw{undef}}] \\
          & \hspace{\sLength}[\kw{tl_caller}\mapsto main][\kw{tl_callee}\mapsto taak2][\kw{x}\mapsto3][\kw{y}\mapsto4][\kw{result}\mapsto 7][\kw{valid}\mapsto \kw{true}]
\end{array}
\)

De waarden van \kw{main_1} en \kw{main_2} zijn inderdaad respectievelijk \(3\) en \(7\), en \kw{grand_total} heeft
inderdaad de waarde \(10\).

\Subsection{Rollback}
Bij een rollback willen we graag bewijzen dat als er een taak is die er niet in slaagt om op juiste wijze tot een
antwoord te komen, dat de hele transactie alsware het een atomair blok teruggaat naar de state waarin de transactie
gestart is.

In het bewijs zal \(taak1\) de waarden \(4\) en \(5\) optellen. Omdat deze onder de grens liggen van \(10\), zal deze
taak vinden dat de antwoorden juist zijn en goed genoeg zijn om door te geven aan de caller. \(taak2\) daarentegen zal
de waarden \(5\) en \(5\) optellen. Hiermee komt \(taak2\) niet door de domeincontrole die ervoor zorgt dat de som niet
boven de \(10\) mag komen. \(taak2\) zegt dat hij niet tot juiste antwoorden heeft kunnen komen, en als reactie daarop
zal \(main\) een rollback in gang zetten.

We verwachten dat de state van \(main\) aan het einde van het programma gelijk is aan de state die \(main\) kreeg na
het spawnen. Ook de taken die betrokken zijn bij de transactie zullen teruggaan naar de state waarmee ze gespawned zijn.

\begin{tikzpicture}
\setcounter{threadnum}{1}

\draw[->,>=angle 60,rounded corners=5pt] (\twidth-2.65\unitlength,5+\lowerhalfbar) %b
									-- (\twidth-2.65\unitlength,.5) % m
									-- (\twidth-2.15\unitlength,.5) % m'
									-- (\twidth-2.15\unitlength,5+\lowerhalfbar); %e

\draw[->,>=angle 60,rounded corners=5pt] (\twidth-1.65\unitlength,5+\lowerhalfbar) %b
									-- (\twidth-1.65\unitlength,.5) % m
									-- (\twidth-1.15\unitlength,.5) % m'
									-- (\twidth-1.15\unitlength,5+\lowerhalfbar); %e

\draw[] (2.5,\thethreadnum*4-1+2.75) % transactional env
				-- (2.5,\thethreadnum*2-2)
				-- (\twidth-.9\unitlength,\thethreadnum*2-2)
				--	(\twidth-.9\unitlength,\thethreadnum*4-1+2.75)-- cycle;
\node at (\twidth-.9\unitlength,\thethreadnum*4-1+2.75) [below left]{\textbf{transactie2}};

% Instance 3
\draw[dotted] (0,      \thethreadnum*2-1)
		-- (\twidth,\thethreadnum*2-1)
		node [midway, right=.8\unitlength, below=0.75em] {};
\path (0,0)+(0,1) node[inststyle] (inst1) {taak2};
\draw[fill=gray!30] (3.15,\thethreadnum*2-1+\upperhalfbar)
				-- (3.15,\thethreadnum*2-1+\lowerhalfbar)
				-- (\twidth-3.1\unitlength,\thethreadnum*2-1+\lowerhalfbar)
				--	(\twidth-3.1\unitlength,\thethreadnum*2-1+\upperhalfbar)-- cycle;
\draw[fill=gray!30] (\twidth-2.15\unitlength,\thethreadnum*2-1+\upperhalfbar)
				-- (\twidth-2.15\unitlength,\thethreadnum*2-1+\lowerhalfbar)
				-- (\twidth-1.65\unitlength,\thethreadnum*2-1+\lowerhalfbar)
				--	(\twidth-1.65\unitlength,\thethreadnum*2-1+\upperhalfbar)-- cycle;

\stepcounter{threadnum}

% Instance 2
\draw[dotted] (0,      \thethreadnum*2-1)
		-- (\twidth,\thethreadnum*2-1)
		node [midway, right=.8\unitlength, below=0.75em] {};
\path (0,1)+(0,2) node[inststyle] (inst3) {taak1};
\draw[fill=gray!30] (3.15,\thethreadnum*2-1+\upperhalfbar)
				-- (3.15,\thethreadnum*2-1+\lowerhalfbar)
				-- (\twidth-3.4\unitlength,\thethreadnum*2-1+\lowerhalfbar)
				--	(\twidth-3.4\unitlength,\thethreadnum*2-1+\upperhalfbar)-- cycle;
\draw[fill=gray!30] (\twidth-2.15\unitlength,\thethreadnum*2-1+\upperhalfbar)
				-- (\twidth-2.15\unitlength,\thethreadnum*2-1+\lowerhalfbar)
				-- (\twidth-1.65\unitlength,\thethreadnum*2-1+\lowerhalfbar)
				--	(\twidth-1.65\unitlength,\thethreadnum*2-1+\upperhalfbar)-- cycle;

\stepcounter{threadnum}

% Thread 1
\draw[dotted] (0,      \thethreadnum*2-1)
		-- (\twidth,\thethreadnum*2-1)
		node [midway, right=.8\unitlength, above=0.75em] {};
\path (0,2)+(0,3) node[inststyle] (inst2) {main};
\draw[fill=gray!30] (2,\thethreadnum*2-1+\upperhalfbar)
				-- (2,\thethreadnum*2-1+\lowerhalfbar)
				-- (\twidth-.4\unitlength,\thethreadnum*2-1+\lowerhalfbar)
				-- (\twidth-.4\unitlength,\thethreadnum*2-1+\upperhalfbar)-- cycle;

\draw[dotted] (2,5)--(2,6.25) node[marker] () {\footnotesize a};
\draw[dotted] (2.5,5)--(2.5,6.25) node[marker] () {\footnotesize b};
\draw[dotted] (2.75,5)--(2.75,6.25) node[marker] () {\footnotesize c};
\draw[dotted] (3,5)--(3,6.25) node[marker] () {\footnotesize d};
\draw[dotted] (\twidth-3.4\unitlength,3)--(\twidth-3.4\unitlength,6.25) node[marker] () {\footnotesize t};
\draw[dotted] (\twidth-3.1\unitlength,1)--(\twidth-3.1\unitlength,6.25) node[marker] () {\footnotesize u};
\draw[dotted] (\twidth-2.65\unitlength,5)--(\twidth-2.65\unitlength,6.25) node[marker] () {\footnotesize v};
\draw[dotted] (\twidth-2.15\unitlength,5)--(\twidth-2.15\unitlength,6.25) node[marker] () {\footnotesize w};
%\draw[dotted] (\twidth-1.65\unitlength,1)--(\twidth-1.65\unitlength,6.25) node[marker] () {\footnotesize x};
%\draw[dotted] (\twidth-1.15\unitlength,3)--(\twidth-1.15\unitlength,6.25) node[marker] () {\footnotesize y};
\draw[dotted] (\twidth-.9\unitlength,5)--(\twidth-.9\unitlength,6.25) node[marker] () {\footnotesize z};

% Result/argument arrows
%t3
\node (b) at (2.75,5) {};
\node (m) at (2.75,1) {};
\node (e) at (3.25,1) {};
\draw[->,>=angle 60] (b) |- (m) -- (e) node {};

%t2
\node (b) at (3,5) {};
\node (m) at (3,3) {};
\node (e) at (3.25,3) {};
\draw[->,>=angle 60] (b) |- (m) -- (e) node {};

\draw[decorate,decoration=zigzag] {(\twidth-3.1\unitlength, 1) -- (\twidth-2.65\unitlength,1)}; %t3
\draw[decorate,decoration=zigzag] {(\twidth-3.4\unitlength, 3) -- (\twidth-2.65\unitlength,3)}; %t2

\end{tikzpicture}

\begin{lstlisting}[caption={Rollback}]
func optellen tl tl vn tid var x:=2; var y:=3; epsilon is
start
	result := x + y;
	if result < 10
	then
		valid := true
	else
		valid := false
	if set_result tl vn result valid tid then
		skip end_commit_transaction tl tid
	else
		skip end_rollback_transaction tl tid
end;
func main tl tl vn tid epsilon is
start
	start_transaction transactie2
	main_1:=undef;
	main_2:=undef;
	spawn fl:=optellen tl_callee:=taak1 tl_caller:=main vn:=main_1 tid:=transactie2
		var x:=4; var y:=5; epsilon;
	spawn fl:=optellen tl_callee:=taak2 tl_caller:=main vn:=main_2 tid:=transactie2
		var x:=5; var y:=5; epsilon;
	if wait taak1 then skip else skip;
	if wait taak2 then skip else skip
	if collect_votes transactie2 then
		grand_total := main_1 + main_2;
		if grand_total < 15 then
			skip;
			commit_transaction transactie2
		else
			skip;
			rollback_transaction transactie2
	else
		grand_total := 0;
		rollback_transaction transactie2
end;
epsilon
\end{lstlisting}

\(
\begin{array}{rl}
& Executieomgeving \\\hline
\kw{func1} & s  \\
\kw{func1} & s  \\
\kw{func2} & s  \\
\kw{spawn} & s  \\
& s \\
& \\
& main\markers{a} \\\hline
\kw{var2} & s[\kw{tl}\mapsto \kw{undef}][\kw{tid}\mapsto \kw{undef}][\kw{vn}\mapsto \kw{undef}]\\
\kw{call} & s[\kw{tl}\mapsto \kw{undef}][\kw{tid}\mapsto \kw{undef}][\kw{vn}\mapsto \kw{undef}]\markers{b}\\
\kw{transaction} & s[\kw{tl}\mapsto \kw{undef}][\kw{tid}\mapsto \kw{undef}][\kw{vn}\mapsto \kw{undef}]\\
\kw{ass} & s[\kw{tl}\mapsto \kw{undef}][\kw{tid}\mapsto transactie2][\kw{vn}\mapsto \kw{undef}]\\
\kw{ass} & s[\kw{tl}\mapsto \kw{undef}][\kw{tid}\mapsto transactie2][\kw{vn}\mapsto \kw{undef}][\kw{main_1}\mapsto \kw{undef}]\markers{c}\\
\kw{spawn} & s[\kw{tl}\mapsto \kw{undef}][\kw{tid}\mapsto transactie2][\kw{vn}\mapsto \kw{undef}][\kw{main_1}\mapsto \kw{undef}][\kw{main_2}\mapsto \kw{undef}]\markers{d}\\
\kw{spawn} & s[\kw{tl}\mapsto \kw{undef}][\kw{tid}\mapsto transactie2][\kw{vn}\mapsto \kw{undef}][\kw{main_1}\mapsto \kw{undef}][\kw{main_2}\mapsto \kw{undef}]\\
\kw{if tt} & s[\kw{tl}\mapsto \kw{undef}][\kw{tid}\mapsto transactie2][\kw{vn}\mapsto \kw{undef}][\kw{main_1}\mapsto \kw{undef}][\kw{main_2}\mapsto \kw{undef}]\\

\kw{skip} & s[\kw{tl}\mapsto \kw{undef}][\kw{tid}\mapsto transactie2][\kw{vn}\mapsto \kw{undef}][\kw{main_1}\mapsto 9][\kw{main_2}\mapsto \kw{undef}]  \\
\kw{if ff} & s[\kw{tl}\mapsto \kw{undef}][\kw{tid}\mapsto transactie2][\kw{vn}\mapsto \kw{undef}][\kw{main_1}\mapsto 9][\kw{main_2}\mapsto \kw{undef}]  \\
\kw{skip} & s[\kw{tl}\mapsto \kw{undef}][\kw{tid}\mapsto transactie2][\kw{vn}\mapsto \kw{undef}][\kw{main_1}\mapsto 9][\kw{main_2}\mapsto \kw{undef}] \markers{v} \\
\kw{if ff} & s[\kw{tl}\mapsto \kw{undef}][\kw{tid}\mapsto transactie2][\kw{vn}\mapsto \kw{undef}][\kw{main_1}\mapsto 9][\kw{main_2}\mapsto \kw{undef}]  \\
\kw{ass} & s[\kw{tl}\mapsto \kw{undef}][\kw{tid}\mapsto transactie2][\kw{vn}\mapsto \kw{undef}][\kw{main_1}\mapsto 9][\kw{main_2}\mapsto \kw{undef}] \markers{w,z} \\
\kw{rollback} & s[\kw{tl}\mapsto \kw{undef}][\kw{tid}\mapsto transactie2][\kw{vn}\mapsto \kw{undef}][\kw{main_1}\mapsto 9][\kw{main_2}\mapsto \kw{undef}] \\
		  & s[\kw{tl}\mapsto \kw{undef}][\kw{tid}\mapsto \kw{undef}][\kw{vn}\mapsto \kw{undef}] \\
& \\
& taak1 \\\hline
\kw{var1} & s[\kw{tl}\mapsto \kw{undef}][\kw{tid}\mapsto transactie2][\kw{vn}\mapsto main_1][\kw{main_1}\mapsto \kw{undef}][\kw{main_2}\mapsto \kw{undef}]  \\
          & \hspace{\sLength}[\kw{tl_caller}\mapsto main][\kw{tl_callee}\mapsto taak1]  \\
\kw{var1} & s[\kw{tl}\mapsto \kw{undef}][\kw{tid}\mapsto transactie2][\kw{vn}\mapsto main_1][\kw{main_1}\mapsto \kw{undef}][\kw{main_2}\mapsto \kw{undef}]  \\
          & \hspace{\sLength}[\kw{tl_caller}\mapsto main][\kw{tl_callee}\mapsto taak1][\kw{x}\mapsto 2]  \\
\kw{var2} & s[\kw{tl}\mapsto \kw{undef}][\kw{tid}\mapsto transactie2][\kw{vn}\mapsto main_1][\kw{main_1}\mapsto \kw{undef}][\kw{main_2}\mapsto \kw{undef}]  \\
          & \hspace{\sLength}[\kw{tl_caller}\mapsto main][\kw{tl_callee}\mapsto taak1][\kw{x}\mapsto 2][\kw{y}\mapsto 3]  \\
\kw{var1} & s[\kw{tl}\mapsto \kw{undef}][\kw{tid}\mapsto transactie2][\kw{vn}\mapsto main_1][\kw{main_1}\mapsto \kw{undef}][\kw{main_2}\mapsto \kw{undef}]  \\
          & \hspace{\sLength}[\kw{tl_caller}\mapsto main][\kw{tl_callee}\mapsto taak1][\kw{x}\mapsto 2][\kw{y}\mapsto 3]  \\
\kw{var1} & s[\kw{tl}\mapsto \kw{undef}][\kw{tid}\mapsto transactie2][\kw{vn}\mapsto main_1][\kw{main_1}\mapsto \kw{undef}][\kw{main_2}\mapsto \kw{undef}]  \\
          & \hspace{\sLength}[\kw{tl_caller}\mapsto main][\kw{tl_callee}\mapsto taak1][\kw{x}\mapsto 4][\kw{y}\mapsto 3]  \\
\kw{var2} & s[\kw{tl}\mapsto \kw{undef}][\kw{tid}\mapsto transactie2][\kw{vn}\mapsto main_1][\kw{main_1}\mapsto \kw{undef}][\kw{main_2}\mapsto \kw{undef}]  \\
          & \hspace{\sLength}[\kw{tl_caller}\mapsto main][\kw{tl_callee}\mapsto taak1][\kw{x}\mapsto 4][\kw{y}\mapsto 5]  \\
\kw{call} & s[\kw{tl}\mapsto \kw{undef}][\kw{tid}\mapsto transactie2][\kw{vn}\mapsto main_1][\kw{main_1}\mapsto \kw{undef}][\kw{main_2}\mapsto \kw{undef}]  \\
          & \hspace{\sLength}[\kw{tl_caller}\mapsto main][\kw{tl_callee}\mapsto taak1][\kw{x}\mapsto 4][\kw{y}\mapsto 5]  \\
\kw{ass} & s[\kw{tl}\mapsto \kw{undef}][\kw{tid}\mapsto transactie2][\kw{vn}\mapsto main_1][\kw{main_1}\mapsto \kw{undef}][\kw{main_2}\mapsto \kw{undef}]  \\
          & \hspace{\sLength}[\kw{tl_caller}\mapsto main][\kw{tl_callee}\mapsto taak1][\kw{x}\mapsto 4][\kw{y}\mapsto 5]  \\
\kw{if tt} & s[\kw{tl}\mapsto \kw{undef}][\kw{tid}\mapsto transactie2][\kw{vn}\mapsto main_1][\kw{main_1}\mapsto \kw{undef}][\kw{main_2}\mapsto \kw{undef}]  \\
          & \hspace{\sLength}[\kw{tl_caller}\mapsto main][\kw{tl_callee}\mapsto taak1][\kw{x}\mapsto 4][\kw{y}\mapsto 5][\kw{result}\mapsto 9]  \\
\kw{ass} & s[\kw{tl}\mapsto \kw{undef}][\kw{tid}\mapsto transactie2][\kw{vn}\mapsto main_1][\kw{main_1}\mapsto \kw{undef}][\kw{main_2}\mapsto \kw{undef}]  \\
          & \hspace{\sLength}[\kw{tl_caller}\mapsto main][\kw{tl_callee}\mapsto taak1][\kw{x}\mapsto 4][\kw{y}\mapsto 5][\kw{result}\mapsto 9]\markers{t}  \\
\kw{if ff} & s[\kw{tl}\mapsto \kw{undef}][\kw{tid}\mapsto transactie2][\kw{vn}\mapsto main_1][\kw{main_1}\mapsto \kw{undef}][\kw{main_2}\mapsto \kw{undef}]  \\
          & \hspace{\sLength}[\kw{tl_caller}\mapsto main][\kw{tl_callee}\mapsto taak1][\kw{x}\mapsto 4][\kw{y}\mapsto 5][\kw{result}\mapsto 9] \\
          & \hspace{\sLength}[\kw{valid}\mapsto \kw{true}]  \\
\kw{skip} & s[\kw{tl}\mapsto \kw{undef}][\kw{tid}\mapsto transactie2][\kw{vn}\mapsto main_1][\kw{main_1}\mapsto \kw{undef}][\kw{main_2}\mapsto \kw{undef}]  \\
          & \hspace{\sLength}[\kw{tl_caller}\mapsto main][\kw{tl_callee}\mapsto taak1][\kw{x}\mapsto 4][\kw{y}\mapsto 5][\kw{result}\mapsto 9] \\
          & \hspace{\sLength}[\kw{valid}\mapsto \kw{true}]  \\
\kw{end_rollback} & s[\kw{tl}\mapsto \kw{undef}][\kw{tid}\mapsto transactie2][\kw{vn}\mapsto main_2][\kw{main_1}\mapsto \kw{undef}][\kw{main_2}\mapsto \kw{undef}] \\
          & \hspace{\sLength}[\kw{tl_caller}\mapsto main][\kw{tl_callee}\mapsto taak1][\kw{x}\mapsto 4][\kw{y}\mapsto 5][\kw{result}\mapsto 9] \\
          & \hspace{\sLength}[\kw{valid}\mapsto \kw{true}] \\
\end{array}
\)

\(
\begin{array}{rl}
& s[\kw{tl}\mapsto \kw{undef}][\kw{tid}\mapsto transactie2][\kw{vn}\mapsto main_2][\kw{main_1}\mapsto \kw{undef}][\kw{main_2}\mapsto \kw{undef}] \\
          & \hspace{\sLength}[\kw{tl_caller}\mapsto main][\kw{tl_callee}\mapsto taak1] \\
& \\
& taak2 \\\hline
\kw{var1} & s[\kw{tl}\mapsto \kw{undef}][\kw{tid}\mapsto transactie2][\kw{vn}\mapsto main_2][\kw{main_1}\mapsto \kw{undef}][\kw{main_2}\mapsto \kw{undef}]  \\
          & \hspace{\sLength}[\kw{tl_caller}\mapsto main][\kw{tl_callee}\mapsto taak2]  \\
\kw{var1} & s[\kw{tl}\mapsto \kw{undef}][\kw{tid}\mapsto transactie2][\kw{vn}\mapsto main_2][\kw{main_1}\mapsto \kw{undef}][\kw{main_2}\mapsto \kw{undef}]  \\
          & \hspace{\sLength}[\kw{tl_caller}\mapsto main][\kw{tl_callee}\mapsto taak2][\kw{x}\mapsto 2]  \\
\kw{var2} & s[\kw{tl}\mapsto \kw{undef}][\kw{tid}\mapsto transactie2][\kw{vn}\mapsto main_2][\kw{main_1}\mapsto \kw{undef}][\kw{main_2}\mapsto \kw{undef}]  \\
          & \hspace{\sLength}[\kw{tl_caller}\mapsto main][\kw{tl_callee}\mapsto taak2][\kw{x}\mapsto 2][\kw{y}\mapsto 3]  \\
\kw{var1} & s[\kw{tl}\mapsto \kw{undef}][\kw{tid}\mapsto transactie2][\kw{vn}\mapsto main_2][\kw{main_1}\mapsto \kw{undef}][\kw{main_2}\mapsto \kw{undef}]  \\
          & \hspace{\sLength}[\kw{tl_caller}\mapsto main][\kw{tl_callee}\mapsto taak2][\kw{x}\mapsto 2][\kw{y}\mapsto 3]  \\
\kw{var1} & s[\kw{tl}\mapsto \kw{undef}][\kw{tid}\mapsto transactie2][\kw{vn}\mapsto main_2][\kw{main_1}\mapsto \kw{undef}][\kw{main_2}\mapsto \kw{undef}]  \\
          & \hspace{\sLength}[\kw{tl_caller}\mapsto main][\kw{tl_callee}\mapsto taak2][\kw{x}\mapsto 5][\kw{y}\mapsto 3]  \\
\kw{var2} & s[\kw{tl}\mapsto \kw{undef}][\kw{tid}\mapsto transactie2][\kw{vn}\mapsto main_2][\kw{main_1}\mapsto \kw{undef}][\kw{main_2}\mapsto \kw{undef}]  \\
          & \hspace{\sLength}[\kw{tl_caller}\mapsto main][\kw{tl_callee}\mapsto taak2][\kw{x}\mapsto 5][\kw{y}\mapsto 5]  \\
\kw{call} & s[\kw{tl}\mapsto \kw{undef}][\kw{tid}\mapsto transactie2][\kw{vn}\mapsto main_2][\kw{main_1}\mapsto \kw{undef}][\kw{main_2}\mapsto \kw{undef}]  \\
          & \hspace{\sLength}[\kw{tl_caller}\mapsto main][\kw{tl_callee}\mapsto taak2][\kw{x}\mapsto 5][\kw{y}\mapsto 5]  \\
\kw{ass} & s[\kw{tl}\mapsto \kw{undef}][\kw{tid}\mapsto transactie2][\kw{vn}\mapsto main_2][\kw{main_1}\mapsto \kw{undef}][\kw{main_2}\mapsto \kw{undef}]  \\
          & \hspace{\sLength}[\kw{tl_caller}\mapsto main][\kw{tl_callee}\mapsto taak2][\kw{x}\mapsto 5][\kw{y}\mapsto 5]  \\
\kw{if ff} & s[\kw{tl}\mapsto \kw{undef}][\kw{tid}\mapsto transactie2][\kw{vn}\mapsto main_2][\kw{main_1}\mapsto \kw{undef}][\kw{main_2}\mapsto \kw{undef}]  \\
          & \hspace{\sLength}[\kw{tl_caller}\mapsto main][\kw{tl_callee}\mapsto taak2][\kw{x}\mapsto 5][\kw{y}\mapsto 5][\kw{result}\mapsto 10]  \\
\kw{ass} & s[\kw{tl}\mapsto \kw{undef}][\kw{tid}\mapsto transactie2][\kw{vn}\mapsto main_2][\kw{main_1}\mapsto \kw{undef}][\kw{main_2}\mapsto \kw{undef}]  \\
          & \hspace{\sLength}[\kw{tl_caller}\mapsto main][\kw{tl_callee}\mapsto taak2][\kw{x}\mapsto 5][\kw{y}\mapsto 5][\kw{result}\mapsto 10] \marker{u} \\
\kw{if ff} & s[\kw{tl}\mapsto \kw{undef}][\kw{tid}\mapsto transactie2][\kw{vn}\mapsto main_2][\kw{main_1}\mapsto \kw{undef}][\kw{main_2}\mapsto \kw{undef}]  \\
          & \hspace{\sLength}[\kw{tl_caller}\mapsto main][\kw{tl_callee}\mapsto taak2][\kw{x}\mapsto 5][\kw{y}\mapsto 5][\kw{result}\mapsto 10] \\
          & \hspace{\sLength}[\kw{valid}\mapsto \kw{false}]  \\
\kw{skip} & s[\kw{tl}\mapsto \kw{undef}][\kw{tid}\mapsto transactie2][\kw{vn}\mapsto main_2][\kw{main_1}\mapsto \kw{undef}][\kw{main_2}\mapsto \kw{undef}]  \\
          & \hspace{\sLength}[\kw{tl_caller}\mapsto main][\kw{tl_callee}\mapsto taak2][\kw{x}\mapsto 5][\kw{y}\mapsto 5][\kw{result}\mapsto 10] \\
          & \hspace{\sLength}[\kw{valid}\mapsto \kw{false}]  \\
\kw{end_rollback} & s[\kw{tl}\mapsto \kw{undef}][\kw{tid}\mapsto transactie2][\kw{vn}\mapsto main_2][\kw{main_1}\mapsto \kw{undef}][\kw{main_2}\mapsto \kw{undef}] \\
          & \hspace{\sLength}[\kw{tl_caller}\mapsto main][\kw{tl_callee}\mapsto taak2][\kw{x}\mapsto 5][\kw{y}\mapsto 5][\kw{result}\mapsto 10] \\
          & \hspace{\sLength}[\kw{valid}\mapsto \kw{false}] \\
& s[\kw{tl}\mapsto \kw{undef}][\kw{tid}\mapsto transactie2][\kw{vn}\mapsto main_2][\kw{main_1}\mapsto \kw{undef}][\kw{main_2}\mapsto \kw{undef}] \\
          & \hspace{\sLength}[\kw{tl_caller}\mapsto main][\kw{tl_callee}\mapsto taak2]
\end{array}
\)

Alle taken, dus \(main\), \(taak1\) en \(taak2\) zijn allemaal terug naar de state waarin ze gespawned zijn omdat er
een rollback in gang gezet was. Dit betekent dat het rollbackmechanisme zijn werk gedaan heeft.

\Subsection{Abort}
Bij de het bewijzen dat het systeem blijft functioneren als er een abort plaats vindt, gaat het om hoe dit waargenomen
wordt, en hoe hier mee omgegaan wordt. Zoals dit werk beschrijft zullen we bij een abort een rollback doen.

Omdat we geen statement voor abort hebben, maar de abort iets is dat van buitenaf komt, is er ook geen speciale notatie
voor. De executieomgeving kan wel waarnemen of er een abort optreedt, vandaar dat er op geanticipeerd kan worden.

We verwachten dat als \(taak2\) een abort krijgt, alle taken behalve de taken die een abort gehad hebben, teruggaan
naar de state zoals die was wanneer ze eenmaal gespawned waren.

\begin{tikzpicture}
\setcounter{threadnum}{1}

\draw[->,>=angle 60,rounded corners=5pt] (\twidth-2.65\unitlength,5+\lowerhalfbar) %b
									-- (\twidth-2.65\unitlength,.5) % m
									-- (\twidth-2.15\unitlength,.5) % m'
									-- (\twidth-2.15\unitlength,5+\lowerhalfbar); %e

\draw[->,>=angle 60,rounded corners=5pt] (\twidth-1.65\unitlength,5+\lowerhalfbar) %b
									-- (\twidth-1.65\unitlength,.5) % m
									-- (\twidth-1.15\unitlength,.5) % m'
									-- (\twidth-1.15\unitlength,5+\lowerhalfbar); %e

\draw[] (2.5,\thethreadnum*4-1+2.75) % transactional env
				-- (2.5,\thethreadnum*2-2)
				-- (\twidth-.9\unitlength,\thethreadnum*2-2)
				--	(\twidth-.9\unitlength,\thethreadnum*4-1+2.75)-- cycle;
\node at (\twidth-.9\unitlength,\thethreadnum*4-1+2.75) [below left]{\textbf{transactie3}};

% Instance 3
\draw[dotted] (0,      \thethreadnum*2-1)
		-- (\twidth,\thethreadnum*2-1)
		node [midway, right=.8\unitlength, below=0.75em] {};
\path (0,0)+(0,1) node[inststyle] (inst1) {taak2};
\draw[fill=gray!30] (3.15,\thethreadnum*2-1+\upperhalfbar)
				-- (3.15,\thethreadnum*2-1+\lowerhalfbar)
				-- (\twidth-3.1\unitlength,\thethreadnum*2-1+\lowerhalfbar)
				--	(\twidth-3.1\unitlength,\thethreadnum*2-1+\upperhalfbar)-- cycle;

\stepcounter{threadnum}

% Instance 2
\draw[dotted] (0,      \thethreadnum*2-1)
		-- (\twidth,\thethreadnum*2-1)
		node [midway, right=.8\unitlength, below=0.75em] {};
\path (0,1)+(0,2) node[inststyle] (inst3) {taak1};
\draw[fill=gray!30] (3.15,\thethreadnum*2-1+\upperhalfbar)
				-- (3.15,\thethreadnum*2-1+\lowerhalfbar)
				-- (\twidth-3.4\unitlength,\thethreadnum*2-1+\lowerhalfbar)
				--	(\twidth-3.4\unitlength,\thethreadnum*2-1+\upperhalfbar)-- cycle;
\draw[fill=gray!30] (\twidth-2.15\unitlength,\thethreadnum*2-1+\upperhalfbar)
				-- (\twidth-2.15\unitlength,\thethreadnum*2-1+\lowerhalfbar)
				-- (\twidth-1.65\unitlength,\thethreadnum*2-1+\lowerhalfbar)
				--	(\twidth-1.65\unitlength,\thethreadnum*2-1+\upperhalfbar)-- cycle;

\stepcounter{threadnum}

% Thread 1
\draw[dotted] (0,      \thethreadnum*2-1)
		-- (\twidth,\thethreadnum*2-1)
		node [midway, right=.8\unitlength, above=0.75em] {};
\path (0,2)+(0,3) node[inststyle] (inst2) {main};
\draw[fill=gray!30] (2,\thethreadnum*2-1+\upperhalfbar)
				-- (2,\thethreadnum*2-1+\lowerhalfbar)
				-- (\twidth-.4\unitlength,\thethreadnum*2-1+\lowerhalfbar)
				-- (\twidth-.4\unitlength,\thethreadnum*2-1+\upperhalfbar)-- cycle;

\draw[dotted] (2,5)--(2,6.25) node[marker] () {\footnotesize a};
\draw[dotted] (2.5,5)--(2.5,6.25) node[marker] () {\footnotesize b};
\draw[dotted] (2.75,5)--(2.75,6.25) node[marker] () {\footnotesize c};
\draw[dotted] (3,5)--(3,6.25) node[marker] () {\footnotesize d};
\draw[dotted] (\twidth-3.4\unitlength,3)--(\twidth-3.4\unitlength,6.25) node[marker] () {\footnotesize t};
\draw[dotted] (\twidth-3.1\unitlength,1)--(\twidth-3.1\unitlength,6.25) node[marker] () {\footnotesize u};
\draw[dotted] (\twidth-2.65\unitlength,5)--(\twidth-2.65\unitlength,6.25) node[marker] () {\footnotesize v};
\draw[dotted] (\twidth-2.15\unitlength,5)--(\twidth-2.15\unitlength,6.25) node[marker] () {\footnotesize w};
%\draw[dotted] (\twidth-1.65\unitlength,1)--(\twidth-1.65\unitlength,6.25) node[marker] () {\footnotesize x};
%\draw[dotted] (\twidth-1.15\unitlength,3)--(\twidth-1.15\unitlength,6.25) node[marker] () {\footnotesize y};
\draw[dotted] (\twidth-.9\unitlength,5)--(\twidth-.9\unitlength,6.25) node[marker] () {\footnotesize z};


% Result/argument arrows
%t3
\node (b) at (2.75,5) {};
\node (m) at (2.75,1) {};
\node (e) at (3.25,1) {};
\draw[->,>=angle 60] (b) |- (m) -- (e) node {};

%t2
\node (b) at (3,5) {};
\node (m) at (3,3) {};
\node (e) at (3.25,3) {};
\draw[->,>=angle 60] (b) |- (m) -- (e) node {};

\draw[decorate,decoration=zigzag] {(\twidth-3.4\unitlength, 3) -- (\twidth-2.65\unitlength,3)}; %t2

\end{tikzpicture}

\begin{lstlisting}[caption={Abort}]
func optellen tl tl vn tid var x:=2; var y:=3; epsilon is
start
	result := x + y;
	if result < 10
	then
		valid := true
	else
		valid := false
	if set_result tl vn result valid tid then
		skip end_commit_transaction tl tid
	else
		skip end_rollback_transaction tl tid
end;
func main tl tl vn tid epsilon is
start
	start_transaction transactie3
	main_1:=undef;
	main_2:=undef;
	spawn fl:=optellen tl_callee:=taak1 tl_caller:=main vn:=main_1 tid:=transactie3
		var x:=1; var y:=2; epsilon;
	spawn fl:=optellen tl_callee:=taak2 tl_caller:=main vn:=main_2 tid:=transactie3
		var x:=3; var y:=4; epsilon;
	if wait taak1 then skip else skip;
	if wait taak2 then skip else skip
	if collect_votes transactie3 then
		grand_total := main_1 + main_2;
		if grand_total < 15 then
			skip;
			commit_transaction transactie3
		else
			skip;
			rollback_transaction transactie3
	else
		grand_total := 0;
		rollback_transaction transactie3
end;
epsilon
\end{lstlisting}

\(
\begin{array}{rl}
& Executieomgeving \\\hline
\kw{func1} & s  \\
\kw{func1} & s  \\
\kw{func2} & s  \\
\kw{spawn} & s  \\
& s \\
& \\
& main\marker{a} \\\hline
\kw{var2} & s[\kw{tl}\mapsto \kw{undef}][\kw{tid}\mapsto \kw{undef}][\kw{vn}\mapsto \kw{undef}]\\
\kw{call} & s[\kw{tl}\mapsto \kw{undef}][\kw{tid}\mapsto \kw{undef}][\kw{vn}\mapsto \kw{undef}]\marker{b}\\
\kw{transaction} & s[\kw{tl}\mapsto \kw{undef}][\kw{tid}\mapsto \kw{undef}][\kw{vn}\mapsto \kw{undef}]\\
\kw{ass} & s[\kw{tl}\mapsto \kw{undef}][\kw{tid}\mapsto transactie3][\kw{vn}\mapsto \kw{undef}]\\
\kw{ass} & s[\kw{tl}\mapsto \kw{undef}][\kw{tid}\mapsto transactie3][\kw{vn}\mapsto \kw{undef}][\kw{main_1}\mapsto \kw{undef}]\marker{c}\\
\kw{spawn} & s[\kw{tl}\mapsto \kw{undef}][\kw{tid}\mapsto transactie3][\kw{vn}\mapsto \kw{undef}][\kw{main_1}\mapsto \kw{undef}][\kw{main_2}\mapsto \kw{undef}]\marker{d}\\
\kw{spawn} & s[\kw{tl}\mapsto \kw{undef}][\kw{tid}\mapsto transactie3][\kw{vn}\mapsto \kw{undef}][\kw{main_1}\mapsto \kw{undef}][\kw{main_2}\mapsto \kw{undef}]\\
\kw{if tt} & s[\kw{tl}\mapsto \kw{undef}][\kw{tid}\mapsto transactie3][\kw{vn}\mapsto \kw{undef}][\kw{main_1}\mapsto \kw{undef}][\kw{main_2}\mapsto \kw{undef}]\\

\kw{skip} & s[\kw{tl}\mapsto \kw{undef}][\kw{tid}\mapsto transactie3][\kw{vn}\mapsto \kw{undef}][\kw{main_1}\mapsto 9][\kw{main_2}\mapsto \kw{undef}]  \\
\kw{if ff} & s[\kw{tl}\mapsto \kw{undef}][\kw{tid}\mapsto transactie3][\kw{vn}\mapsto \kw{undef}][\kw{main_1}\mapsto 9][\kw{main_2}\mapsto \kw{undef}]  \\
\kw{skip} & s[\kw{tl}\mapsto \kw{undef}][\kw{tid}\mapsto transactie3][\kw{vn}\mapsto \kw{undef}][\kw{main_1}\mapsto 9][\kw{main_2}\mapsto \kw{undef}] \marker{v}  \\
\kw{if ff} & s[\kw{tl}\mapsto \kw{undef}][\kw{tid}\mapsto transactie3][\kw{vn}\mapsto \kw{undef}][\kw{main_1}\mapsto 9][\kw{main_2}\mapsto \kw{undef}]  \\
\kw{ass} & s[\kw{tl}\mapsto \kw{undef}][\kw{tid}\mapsto transactie3][\kw{vn}\mapsto \kw{undef}][\kw{main_1}\mapsto 9][\kw{main_2}\mapsto \kw{undef}] \marker{w,z} \\
\kw{rollback} & s[\kw{tl}\mapsto \kw{undef}][\kw{tid}\mapsto transactie3][\kw{vn}\mapsto \kw{undef}][\kw{main_1}\mapsto 9][\kw{main_2}\mapsto \kw{undef}] \\
& s[\kw{tl}\mapsto \kw{undef}][\kw{tid}\mapsto \kw{undef}][\kw{vn}\mapsto \kw{undef}] \\
& \\
& taak1 \\\hline
\kw{var1} & s[\kw{tl}\mapsto \kw{undef}][\kw{tid}\mapsto transactie3][\kw{vn}\mapsto main_1][\kw{main_1}\mapsto \kw{undef}][\kw{main_2}\mapsto \kw{undef}]  \\
          & \hspace{\sLength}[\kw{tl_caller}\mapsto main][\kw{tl_callee}\mapsto taak1]  \\
\kw{var1} & s[\kw{tl}\mapsto \kw{undef}][\kw{tid}\mapsto transactie3][\kw{vn}\mapsto main_1][\kw{main_1}\mapsto \kw{undef}][\kw{main_2}\mapsto \kw{undef}]  \\
          & \hspace{\sLength}[\kw{tl_caller}\mapsto main][\kw{tl_callee}\mapsto taak1][\kw{x}\mapsto 2]  \\
\kw{var2} & s[\kw{tl}\mapsto \kw{undef}][\kw{tid}\mapsto transactie3][\kw{vn}\mapsto main_1][\kw{main_1}\mapsto \kw{undef}][\kw{main_2}\mapsto \kw{undef}]  \\
          & \hspace{\sLength}[\kw{tl_caller}\mapsto main][\kw{tl_callee}\mapsto taak1[\kw{x}\mapsto 2][\kw{y}\mapsto 3]  \\
\kw{var1} & s[\kw{tl}\mapsto \kw{undef}][\kw{tid}\mapsto transactie3][\kw{vn}\mapsto main_1][\kw{main_1}\mapsto \kw{undef}][\kw{main_2}\mapsto \kw{undef}]  \\
          & \hspace{\sLength}[\kw{tl_caller}\mapsto main][\kw{tl_callee}\mapsto taak1]  \\
\kw{var1} & s[\kw{tl}\mapsto \kw{undef}][\kw{tid}\mapsto transactie3][\kw{vn}\mapsto main_1][\kw{main_1}\mapsto \kw{undef}][\kw{main_2}\mapsto \kw{undef}]  \\
          & \hspace{\sLength}[\kw{tl_caller}\mapsto main][\kw{tl_callee}\mapsto taak1][\kw{x}\mapsto 4]  \\
\kw{var2} & s[\kw{tl}\mapsto \kw{undef}][\kw{tid}\mapsto transactie3][\kw{vn}\mapsto main_1][\kw{main_1}\mapsto \kw{undef}][\kw{main_2}\mapsto \kw{undef}]  \\
          & \hspace{\sLength}[\kw{tl_caller}\mapsto main][\kw{tl_callee}\mapsto taak1][\kw{x}\mapsto 4][\kw{y}\mapsto 5]  \\
\kw{call} & s[\kw{tl}\mapsto \kw{undef}][\kw{tid}\mapsto transactie3][\kw{vn}\mapsto main_1][\kw{main_1}\mapsto \kw{undef}][\kw{main_2}\mapsto \kw{undef}]  \\
          & \hspace{\sLength}[\kw{tl_caller}\mapsto main][\kw{tl_callee}\mapsto taak1][\kw{x}\mapsto 4][\kw{y}\mapsto 5]  \\
\kw{ass} & s[\kw{tl}\mapsto \kw{undef}][\kw{tid}\mapsto transactie3][\kw{vn}\mapsto main_1][\kw{main_1}\mapsto \kw{undef}][\kw{main_2}\mapsto \kw{undef}]  \\
          & \hspace{\sLength}[\kw{tl_caller}\mapsto main][\kw{tl_callee}\mapsto taak1][\kw{x}\mapsto 4][\kw{y}\mapsto 5]  \\
\kw{if tt} & s[\kw{tl}\mapsto \kw{undef}][\kw{tid}\mapsto transactie3][\kw{vn}\mapsto main_1][\kw{main_1}\mapsto \kw{undef}][\kw{main_2}\mapsto \kw{undef}]  \\
          & \hspace{\sLength}[\kw{tl_caller}\mapsto main][\kw{tl_callee}\mapsto taak1][\kw{x}\mapsto 4][\kw{y}\mapsto 5][\kw{result}\mapsto 9]  \\
\kw{ass} & s[\kw{tl}\mapsto \kw{undef}][\kw{tid}\mapsto transactie3][\kw{vn}\mapsto main_1][\kw{main_1}\mapsto \kw{undef}][\kw{main_2}\mapsto \kw{undef}]  \\
          & \hspace{\sLength}[\kw{tl_caller}\mapsto main][\kw{tl_callee}\mapsto taak1][\kw{x}\mapsto 4][\kw{y}\mapsto 5][\kw{result}\mapsto 9] \marker{t}  \\
\kw{if ff} & s[\kw{tl}\mapsto \kw{undef}][\kw{tid}\mapsto transactie3][\kw{vn}\mapsto main_1][\kw{main_1}\mapsto \kw{undef}][\kw{main_2}\mapsto \kw{undef}]  \\
          & \hspace{\sLength}[\kw{tl_caller}\mapsto main][\kw{tl_callee}\mapsto taak1][\kw{x}\mapsto 4][\kw{y}\mapsto 5][\kw{result}\mapsto 9] \\
          & \hspace{\sLength}[\kw{valid}\mapsto \kw{true}]  \\
\kw{skip} & s[\kw{tl}\mapsto \kw{undef}][\kw{tid}\mapsto transactie3][\kw{vn}\mapsto main_1][\kw{main_1}\mapsto \kw{undef}][\kw{main_2}\mapsto \kw{undef}]  \\
          & \hspace{\sLength}[\kw{tl_caller}\mapsto main][\kw{tl_callee}\mapsto taak1][\kw{x}\mapsto 4][\kw{y}\mapsto 5][\kw{result}\mapsto 9] \\
          & \hspace{\sLength}[\kw{valid}\mapsto \kw{true}]  \\
\kw{end_rollback} & s[\kw{tl}\mapsto \kw{undef}][\kw{tid}\mapsto transactie3][\kw{vn}\mapsto main_2][\kw{main_1}\mapsto \kw{undef}][\kw{main_2}\mapsto \kw{undef}] \\
          & \hspace{\sLength}[\kw{tl_caller}\mapsto main][\kw{tl_callee}\mapsto taak1][\kw{x}\mapsto 4][\kw{y}\mapsto 5][\kw{result}\mapsto 9] \\
          & \hspace{\sLength}[\kw{valid}\mapsto \kw{true}] \\
\end{array}
\)

\(
\begin{array}{rl}
		  & s[\kw{tl}\mapsto \kw{undef}][\kw{tid}\mapsto transactie3][\kw{vn}\mapsto main_2][\kw{main_1}\mapsto \kw{undef}][\kw{main_2}\mapsto \kw{undef}] \\
          & \hspace{\sLength}[\kw{tl_caller}\mapsto main][\kw{tl_callee}\mapsto taak1] \\
& \\
& taak2 \\\hline
\kw{var1} & s[\kw{tl}\mapsto \kw{undef}][\kw{tid}\mapsto transactie3][\kw{vn}\mapsto main_2][\kw{main_1}\mapsto \kw{undef}][\kw{main_2}\mapsto \kw{undef}]  \\
          & \hspace{\sLength}[\kw{tl_caller}\mapsto main][\kw{tl_callee}\mapsto taak2] \\
          & \textrm{De taak be\"eindigt om onbekende redenen}
\end{array}
\)


\(taak1\) en \(main\) zijn inderdaad teruggegaan naar de startstate en daarmee is bewezen dat wanneer een abort
optreedt in een taak, deze door het het systeem goed afgehandeld wordt.
