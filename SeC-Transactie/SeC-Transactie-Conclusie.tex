\section*{Conclusie}

\Subsection{Bespreking}

In het voorgaand hebben we een aantal uitbreidingen aan de taal While toegevoegd. De kern van de uitbreidingen bestaat
uit uitbreidingen om transacties in parallelle taken uit te voeren.

Wij hebben er voor gekozen om de uitbredingen minimaal te houden. Dat betekent dat een aantal gevallen die in ``echte
parallelle transacties'' voorkomen niet met behulp van deze uitbreidingen afgehandeld kunnen worden.

Het is nu mogelijk dat het programma nooit uit een \kw{wait} toestand komt: als een taak een \kw{while} \kw{true}
\kw{skip} uitvoert zonder ooit \kw{set_result} aan te roepen zal een caller die \kw{wait} aanroept voor deze callee
nooit uit de blokkerende toestand van \kw{wait} komen.

Het niet ontwaken uit een blokkerende toestand zou verholpen kunnen worden door het invoeren van timers: als een taak
langer dan $n$-tijd geblokkerd is dan returnt de functie met een speciale waarde waarna het programma weer verder kan
gaan. Om capaciteitsredenen is een dergelijke uitbreiding buiten beschouwing gelaten.

Het transactiemechanisme, in essentie een 2-phase commit implementatie, is kwetsbaar voor het aborten van de
taak die de transactie is begonnen: als die taak tussen de \kw{collect_votes} en de \kw{commit_transaction} abort dan
zullen de onderliggende taken niet uit de blokkerende toestand van \kw{set_result} komen. Een vergelijkbaar opmerking
kan gemaakt worden over \kw{commit/rollback_transaction} en \kw{end_commit/rollback_transaction}, maar dan voor een
abort van een van de onderliggende taken. Dit is een bekend probleem voor 2-phase commit in het algemeen: zie
\cite{SkeenStonebraker}.


