In het werk zal voortdurend gebruik gemaakt worden van het begrip programma. Een programma in essentie niets anders dan
een verzameling van functies (zie~\ref{sec:Functies}). Deze functies zijn op hun beurt weer een sequentie van
statements.

Een programma in While is ook een serie statements, maar hier worden verder geen eisen aan gesteld;
\(\mathcal{S}|[z:=x|](x\mapsto5)\) kan ook al een geldig programma zijn. Bij ons is een programma hetgeen de programmeur
opschrijft (zie~\ref{sec:Programma}). Dus programma als begrip omvat alles wat minimaal ge\"eist wordt door de
specificaties, en dit omvat ook het begrip programma uit While, de sequentie van statements. In deze programma's mag
gebruik gemaakt worden van alle mogelijkheden die While biedt, maar nu zijn er ook een aantal uitbreidingen die gebruikt
mogen worden zoals het opstarten van parallelle taken en het opstarten van een transactionele omgeving met alle voor- en
nadelen die ermee verbonden zijn.

\Section{Executieomgeving}

Een programma staat meestal niet op zich. In de meeste gevallen wordt een programma opgestart binnen een
besturingssysteem. Een dergelijke omgeving zorgt ervoor dat allerlei infrastructuur niet door ieder programma zelf
ge\"implementeerd hoeft te worden. In een aantal gevallen zullen we omwille van abstractie gebruik maken van een
dergelijke omgeving, hier \Index{executieomgeving} genoemd.

Het is de taak van de executieomgeving om het programma op te starten en alle benodigde variabelen te initialiseren.
Verder zorgt de executieomgeving voor het beheer van de taken en zorgt ervoor dat communicatie tussen de taken mogelijk
is.

\Subsection{Globaliteit}
De executieomgeving is de meest globale omgeving. Dat wilt zeggen dat de executieomgeving alles kan overzien wat er
gebeurt in het programma; het heeft dus zicht op alle taken, variabelen en de algemene progressie.

Omdat onze taal het niet toestaat dat een taak variabelen aanspreekt of wijzigt buiten de omgeving van de taak, levert
dit in gevallen wanneer de ene taak bijvoorbeeld een resultaat terug wilt geven aan een andere taak, problemen op. Om
dit toch goed te laten verlopen wordt er voor de interactie tussen de twee taken gebruik gemaakt van de
executieomgeving als tussenpersoon, zodat de overdracht van gegevens niet alleen goed verloopt, zonder de regels te
overtreden, maar ook om deze overdracht als atomaire stap te beschouwen.

Een andere belangrijke taak van de executieomgeving is het in de gaten houden van de taken. Zeker in het geval van
interactie tussen taken (zoals die in transacties bijvoorbeeld plaats zullen gaan vinden) is het erg handig om te weten
of de taak met wie je wilt samenwerken \"uberhaupt nog wel bestaat. De executieomgeving zal niet per se publiekelijk
maken dat een taak gestorven is, maar wanneer er naar gevraagd wordt kan er onder alle omstandigheden een antwoord
verkregen worden betreffende de status van de taak.

In sommige gevallen maakt de executieomgeving ook bepaalde dingen publiekelijk, dit geldt voor de zogenaamde labels. Op
elk moment is voor elke taak bekend welke functies er allemaal bestaan, maar ook welke andere taken er zijn en welke
transacties er allemaal bezig zijn. Labels die expliciet lokaal zijn binnen de taak zijn de variabelen, dit onder
andere om te verzekeren dat data-integriteit gegarandeerd kan worden (met als uitzondering de executieomgeving, die deze
variabelen wel kan inzien).

Het is voor de individuele taken binnen een programma niet mogelijk om onderling direct te communiceren. Eventuele
communicatie (zoals het wachten op elkaar en het doorgeven van resultaten etc) gebeurt door dit aan de executieomgeving
``te vragen'' door middel van functieaanroepen/systemcalls. Alle communicatie tussen de executieomgeving en individuele
taken gebeurt atomair: er is nooit sprake van enig raceconditie als gevolg van deze communicatie.

De state van de executieomgeving is vergelijkbaar met een environment (zoals in Proc). We kennen hier geen speciale
naam aan toe, maar we gaan er vanuit dat deze omgeving er gewoon is. Op die manier kunnen we er ook voor zorgen dat
taken niet zomaar globale waarden kunnen aanpassen, en dus kan er voor gezorgd worden dat de globale omgeving schoon
blijft van onnodige variabelen. We houden de globale omgeving niet expliciet bij, maar waar nodig komt het in de
semantiek tot uiting dat er iets aangepast wordt.

\Subsection[Undefinedreferences]{Ongedefinieerde variabelen}
Een variabele die \kw{\Index{undef}} is, kan alleen gebruikt worden om te testen of om deze variabel een waarde te
geven. Alle andere bewerkingen met een dergelijk variabele leidt tot een abort van het gehele programma.

De executieomgeving houdt bij welke variabelen en welke labels er op een gegeven moment bestaan. De
executieomgeving kan dus in het geval van een verwijzing naar een label of variabele die niet (meer) bestaat er voor
zorgen dat het programma ophoudt.

Hier zijn ook andere keuzen te maken, maar die zijn niet per se makkelijker als het gaat over het redeneren over het
verloop en de uitkomst van een programma:
\begin{description}
\item[Ongedefinieerd gedrag] Dit is iets wat in de praktijk makkelijker is dan in een formele, speciaal om te bewijzen,
opgezette taal. Dit gedrag is overigens niet handig en zeer onwenselijk, aangezien het programma na zo een ongeldige
aanroep elke uitkomst kan geven. Dit omvat uitkomsten die per definitie fout zijn, maar ook antwoorden die wel eens
goed zouden kunnen zijn. Kortom, aan het antwoord is geen waarde toe te kennen, aangezien deze onder ongedefinieerd
gedrag tot stand gekomen is
\item[Exceptie] Een andere methode om dit soort problemen op te vangen is het gooien van een exceptie. Met behulp van
een exceptie weet je dat er iets mis gegaan is, en kan er naar gehandeld worden.

Dit is een zeer vruchtbaar alternatief voor onze gekozen oplossing; deze uitbreiding vergt echter nog meer
uitbreidingen op While bovenop de al door ons voorgestelde uitbreidingen
\end{description}
