\Section{Semantiek}

% \inference[$TR_{commit}$]{ ST id S; action cv; if action commit esle rollback, s ->{co} s' }{niets -> niets}
% \inference[$TR_{rollback}$]{ ST id S; action cv; $if(action)$ commit $esle$ rollback, s ->{ro} s' }{niets -> niets}

\Subsection{Parallelliteitsoperator}
Om het concept van parallelle taken in While te kunnen introduceren moet er een symbool bedacht worden om dit aan te
geven. Er is gekozen om gebruik te maken van \(\parallel\). Dit symbool geeft aan dat er parallelle paden van executie
plaatsvinden. Anders dan in de While-uitbreiding met parallelisme werken taken echt parallel zonder de notie van
interleaving.

Het programma gaat verder met de state die ontstaan is door \kw{spawn}. \[ \gamma \parallel \gamma' \]

\(\gamma\) is de configuratie waarmee het programma verder gaat, dit is dus altijd een state. \(\gamma'\) is van de vorm
\(\langle S,\:s\rangle\); \(\gamma'\) is dus de parallelle taak.

\Subsection{Undef}
\kw{\Index{undef}} is een speciale waarde die aangeeft dat de waarde van de variabele buiten het domein. Zo is
bijvoorbeeld de waarde van de returnvalue van een functie nog niet bekend, terwijl de variabele al wel gedeclareerd moet
zijn.

\kw{undef} kan gebruikt voor de volgende situaties:
\begin{itemize}
\item Variabelen die \kw{undef} zijn kunnen voor niets anders gebruikt worden dan het testen op het zijn van \kw{undef} en het
voorzien van een waarde
\item Elk ander gebruik van een variabele die \kw{undef} is leidt tot een abort
\end{itemize}

\Subsection[Assignmentsvanbooleans]{Uitbreiding van booleans}
In de syntax zijn een viertal booleaanse expressies toegevoegd: \kw{wait}, \kw{set_result} (met en zonder $tid$) en
\kw{collect_votes}. Hoewel deze expressies sterke gelijkenissen vertonen met statements of functies, is er voor gekozen om ze van het type
booleaanse expressie te maken omdat ze niets wijzigen aan de state van de caller.

Omdat de state van een taak opzichzelf er niet toe doet voor de waarde die \kw{set_result}, \kw{wait} of
\kw{collect\_votes} toegewezen krijgt, moet er een semantiek zijn die de semantiek uit While uitbreidt, maar wel een
constructie toestaat om als het ware te abstraheren van de state die op moment dat de functie wordt aangeroepen geldt
voor die taak. Dit hebben we opgelost door een functie \(\mathcal{B'}\) te introduceren. Deze functie \(\mathcal{B'}\)
is alleen gedefinieerd voor de drie eerdergenoemde functies.

Voor de standaardmanier van het converteren naar een boolean wordt de functie \(\mathcal{B}\) gebruikt, en voor de drie
functies levert deze een aanroep aan naar de nieuwe functie \(\mathcal{B'}\) aan met hetzelfde argument:
\begin{align*}
\mathcal{B}|[\kw{set_result}\:tl\:fl\:x\:a\:b|] s &=
\begin{cases}
\kw{tt}  & \text{als }\mathcal{B'}|[\kw{set_result}\:tl\:fl\:x\:a\:b|] = \kw{true}\\
\kw{ff} & \text{als }\mathcal{B'}|[\kw{set_result}\:tl\:fl\:x\:a\:b|] = \kw{false}\\
\end{cases} \\
\mathcal{B}|[\kw{set_result}\:tl\:fl\:x\:a\:b\:tid|] s &=
\begin{cases}
\kw{tt}  & \text{als }\mathcal{B'}|[\kw{set_result}\:tl\:fl\:x\:a\:b\:tid|] = \kw{true}\\
\kw{ff} & \text{als }\mathcal{B'}|[\kw{set_result}\:tl\:fl\:x\:a\:b\:tid|] = \kw{false}\\
\end{cases} \\
\mathcal{B}|[\kw{wait}\:tl|] s &=
\begin{cases}
\kw{tt}  & \text{als }\mathcal{B'}|[\kw{wait}\:tl|] = \kw{true}\\
\kw{ff} & \text{als }\mathcal{B'}|[\kw{wait}\:tl|] = \kw{false}\\
\end{cases} \\
\mathcal{B}|[\textnormal{\kw{collect_votes}}\:tid|] s &=
\begin{cases}
\kw{tt}  & \text{als }\mathcal{B'}|[\textnormal{\kw{collect_votes}}\:tid|] = \kw{true}\\
\kw{ff} & \text{als }\mathcal{B'}|[\textnormal{\kw{collect_votes}}\:tid|] = \kw{false}\\
\end{cases} \\
\end{align*}
De executieomgeving zal nu, aan de hand van alles wat bij de executieomgeving bekend is, een waarde
toekennen aan de functie zodanig dat:
\begin{align*}
\mathcal{B'}|[\kw{set_result}\:tl\:fl\:x\:a\:b|] &=
\begin{cases}
\kw{true}  & \text{als $b$ = \kw{true}}\\
\kw{false} & \text{als $b$ = \kw{false}}\\
\end{cases} \\
\mathcal{B'}|[\kw{set_result}\:tl\:fl\:x\:a\:b\:tid|] &=
\begin{cases}
\kw{true}  & \text{als commit_transaction is aangeroepen}\\
\kw{false} & \text{als rollback_transaction is aangeroepen}\\
\kw{true}  & \text{anders}
\end{cases} \\
\mathcal{B'}|[\kw{wait}\:tl|] &=
\begin{cases}
\kw{true}  & \text{als set_result van \(tl\) gezet is met \kw{true}}\\
\kw{false} & \text{anders}
\end{cases} \\
\mathcal{B'}|[\textnormal{\kw{collect_votes}}\:tid|] &=
\begin{cases}
\kw{true}  & \text{als set_result van alle \(tl\) binnen \(tid\) gezet is met \kw{true}} \\
\kw{false} & \text{anders}
\end{cases}
\end{align*}

\Subsection{Hulpfuncties}

Op divers plaatsen maken we gebruik van lokale en globale lijsten van variabelen. Om deze lijsten te kunnen
administreren maken we gebruik van diverse hulp functies om elementen toe te voegen, op te vragen en te verwijderen.

\Subsubsection{FunctieLabel}

\(\mathrm{GLOBFL}(FL)\) is de globale lijst van functies  hun (default) \(D_V\) en bijbehorende statements.

Om een functie toe te voegen aan de globale lijst met functies gebruiken we de functie \kw{GLOBFL}.

De semantiek van de functie is:
\begin{eqnarray*}
\left[\kw{GLOBFL}\right]&&
\langle\kw{GLOBFL}\:\kw{func}\:fl\:D_V\:S \:;\:D_F,\:s\rangle => s\\
\end{eqnarray*}

Het toevoegen van een functie gaat als volgt:
\begin{eqnarray*}
	\kw{GLOBFL}(\kw{func}\:fl\:D_V\:S \:;\:D_F) &=& \{\{fl,D_V,S\}\}\cup\mathrm{GLOBFL}(FL) \\
	\kw{GLOBFL}(\epsilon) &=& \mathrm{GLOBFL}(FL)
\end{eqnarray*}

\Subsubsection{TaskLabel}

Er worden twee soorten lijsten van taaklabels bijgehouden. De executieomgeving houdt een globale lijst van alle
taaklabels bij terwijl een taak een lokale lijst met gespawnde taken bijhoudt.

\(\mathrm{GLOBTL}(TL)\) is de globale lijst van taaklabels.

Om een taaklabel toe te voegen aan de globale lijst met taaklabels van alle taken gebruiken we de functie \kw{GLOBTL}.

De semantiek van de functie is:
\begin{eqnarray*}
\left[\kw{GLOBTL}\right]&&
\langle\kw{GLOBTL}\:tl,\:s\rangle => s\\
\end{eqnarray*}

Het toevoegen gaat als volgt:
\begin{eqnarray*}
	\kw{GLOBTL}(tl) &=& \{tl\}\cup\mathrm{GLOBTL}(TL) \\
\end{eqnarray*}

Om een taaklabel te verwijderen uit de globale lijst met taaklabels van alle taken gebruiken we de functie
\kw{GLOBTL'}.

De semantiek van de functie is:
\begin{eqnarray*}
\left[\kw{GLOBTL'}\right]&&
\langle\kw{GLOBTL'}\:tl,\:s\rangle => s\\
\end{eqnarray*}

Het verwijderen gaat als volgt:
\begin{eqnarray*}
	\kw{GLOBTL'}(tl) &=& {GLOBTL}(TL) \setminus \{tl\} \\
\end{eqnarray*}

\(\mathrm{LOCTL}(TL)\) is de lokale lijst van gespawnde taken.

Om een taaklabel toe te voegen aan de lokale lijst met taaklabels met gespawnde taken gebruiken we de functie
\kw{LOCTL}.

De semantiek van de functie is:
\begin{eqnarray*}
\left[\kw{LOCTL}\right]&&
\langle\kw{LOCTL}\:tl,\:s\rangle => s'\\
\end{eqnarray*}

Het toevoegen gaat als volgt:
\begin{eqnarray*}
	\kw{LOCTL}(tl)\:s &=& \{tl\}\cup\mathrm{LOCTL}(TL) \\
\end{eqnarray*}

Om een taaklabel te verwijderen uit de lokale lijst met taaklabels met gespawnde taken gebruiken we de functie
\kw{LOCTL'}.

De semantiek van de functie is:
\begin{eqnarray*}
\left[\kw{LOCTL'}\right]&&
\langle\kw{LOCTL'}\:tl,\:s\rangle => s'\\
\end{eqnarray*}

Het verwijderen gaat als volgt:
\begin{eqnarray*}
	\kw{LOCTL'}(tl)\:s &=& {LOCTL}(TL) \setminus \{tl\} \\
\end{eqnarray*}

\Subsubsection{TransactionLabel}

Er worden twee soorten lijsten van transactielabels bijgehouden. De executieomgeving houdt een globale lijst van
alle transactielabels bij terwijl een taak een lokale lijst met transactielabels van de taak zelf bijhoudt.

\(\mathrm{GLOBTID}(TID)\) is de globale lijst van transactielabels.

Om de transactielabel toe te voegen aan de globale lijst met transactielabel gebruiken we de functie \kw{GLOBTID}.

De semantiek van de functie is:
\begin{eqnarray*}
\left[\kw{GLOBTID}\right]&&
\langle\kw{GLOBTID}\:tid,\:s\rangle => s
\end{eqnarray*}

Het toevoegen gaat als volgt:
\begin{eqnarray*}
	\kw{GLOBTID}(tid) &=& \{tid\}\cup\mathrm{GLOBTID}(TID) \\
\end{eqnarray*}

Om de transactielabel te verwijderen uit de globale lijst met transactielabel gebruiken we de functie
\kw{GLOBTID'}.

De semantiek van de functie is:
\begin{eqnarray*}
\left[\kw{GLOBTID'}\right]&&
\langle\kw{GLOBTID'}\:tid,\:s\rangle => s
\end{eqnarray*}

Het verwijderen gaat als volgt:
\begin{eqnarray*}
	\kw{GLOBTID'}(tid) &=& \kw{GLOBTID}(TID) \setminus \{tid\} \\
\end{eqnarray*}


\(\mathrm{LOCTID}(TID)\) is de lokale lijst van transactielabels.

Om de transactielabel toe te voegen aan de lokale lijst met transactielabel gebruiken we de functie \kw{LOCTID}.

De semantiek van de functie is:
\begin{eqnarray*}
\left[\kw{LOCTID}\right]&&
\langle\kw{LOCTID}\:tid,\:s\rangle => s'
\end{eqnarray*}

Het toevoegen gaat als volgt:
\begin{eqnarray*}
	\kw{LOCTID}(tid) &=& \{tid\}\cup\mathrm{LOCTID}(TID) \\
\end{eqnarray*}

Om de transactielabel te verwijderen uit de lokale lijst met transactielabel gebruiken we de functie
\kw{LOCTID'}.

De semantiek van de functie is:
\begin{eqnarray*}
\left[\kw{LOCTID'}\right]&&
\langle\kw{LOCTID'}\:tid,\:s\rangle => s'
\end{eqnarray*}

Het verwijderen gaat als volgt:
\begin{eqnarray*}
	\kw{LOCTID'}(tid) &=& \mathrm{LOCTID}(TID) \setminus \{tid\} \\
\end{eqnarray*}

%\Subsubsection{VariableName}

\Subsubsection{State}

\(\mathrm{GLOBSTATE}(STATE)\) is de globale lijst van states.

Om een state toe te voegen aan de globale lijst met states gebruiken we de functie \kw{GLOBSTATE}.

De semantiek van de
functie is:
\begin{eqnarray*}
\left[\kw{GLOBSTATE}\right]&&
\langle\kw{GLOBSTATE}\:tl\:tid\:s',\:s\rangle => s
\end{eqnarray*}

Het toevoegen gaat als volgt:
\begin{eqnarray*}
	\kw{GLOBSTATE}(tl\:tid,\:s) &=& \{tl,tid,s\}\cup\mathrm{GLOBSTATE}(STATE) \\
\end{eqnarray*}

Om een state op te vragen uit de globale lijst met states gebruiken we de functie
\kw{GLOBSTATE''}.

De semantiek van de functie is:
\begin{eqnarray*}
\left[\kw{GLOBSTATE''}\right]&&
\langle\kw{GLOBSTATE''}\:tl\:tid\rangle => s
\end{eqnarray*}

Het opvragen gaat als volgt:
\begin{eqnarray*}
	\kw{GLOBSTATE''}(tl\:tid) &=& \{tl,tid,s\} \\
\end{eqnarray*}

Om een state toe te verwijderen uit de globale lijst met states gebruiken we de functie
\kw{GLOBSTATE'}.

De semantiek van de functie is:
\begin{eqnarray*}
\left[\kw{GLOBSTATE'}\right]&&
\langle\kw{GLOBSTATE'}\:tl\:tid,\:s\rangle => s
\end{eqnarray*}

Het verwijderen gaat als volgt:
\begin{eqnarray*}
	\kw{GLOBSTATE'}(tl\:tid) &=& \kw{GLOBSTATE}(STATE) \setminus \{tl,tid,s\} \quad \textrm{Waar \(s\) = \kw{GLOBSTATE''}
	(\(tl\) \(tid\))}\\
\end{eqnarray*}

\Subsection{Statements}

We nemen de bestaande statements uit While over en voegen daar onze statements aan toe. Vanuit While nemen we over:
\begin{eqnarray*}
\left[\kw{ass}\right]&& \langle x:=a.\:s\rangle=>s[x\mapsto\mathcal{A}|[a|]s] \\
\left[\kw{skip}\right]&& \osstep{}{\kw{skip},\:s}{s} \\
\left[\kw{comp}^1\right]&& \inference{\osstep{}{S_1,\:s}{S'_1,\:s'}}{\osstep{}{S_1\:;\:S_2,\:s}{S'_1\:S_2,\:s'}}\\
\left[\kw{comp}^2\right]&& \inference{\osstep{}{S_1,\:s}{s'}}{\osstep{}{S_1\:;\:S_2,\:s}{S_2,\:s'}}\\
\left[\kw{if}^{\kw{true}}\right]&& \osstep{}{\kw{if}\:b\:\kw{then}\:S_1\:\kw{else}\:S_2,\:s}{S_1,\:s}\:\text{ Als
\(\mathcal{B}|[b|]s = \kw{tt}\)} \\
\left[\kw{if}^{\kw{false}}\right]&&\osstep{}{\kw{if}\:b\:\kw{then}\:S_1\:\kw{else}\:S_2,\:s}{S_2,\:s}\:\text{ Als
\(\mathcal{B}|[b|]s = \kw{ff}\)} \\
\left[\kw{while}\right]&&
\osstep{}{\kw{while}\:b\:\kw{do}\:S,\:s}{\kw{if}\:b\:\kw{then}\:(S\:;\:\kw{while}\:b\:\kw{do}\:S)\:\kw{else}\:\kw{skip},\:s}
\end{eqnarray*}
En we voegen daar de statements uit de volgende paragrafen aan toe.

\Subsubsection{$D_V$ en var}
\(D_V\) is een lijst met variabelen, waarbij de variabelen gescheiden worden door een ``;''. De lijst wordt afgesloten
door middel van een \(\epsilon\) (de lege variabele).
\begin{eqnarray*}
\left[\kw{var}^1\right]&&
\osstep{}
{\kw{var}\:x:= a\:;\:D_V,\:s[x\mapsto\mathcal{A}|[a|]s]}
{D_V,\:s'}%
\\%
\left[\kw{var}^2\right]&&
\osstep{}
{\epsilon,\:s}
{s}
\end{eqnarray*}

Wanneer we \(\kw{var}\:x:= a\) aanroepen, dan kijken we of \(x\) al bestaat in de huidige state:
\begin{itemize}
  \item Wanneer \(x\) nog niet bestaat, dan wordt er een variabele \(x\) ge\"introduceerd met de waarde \(a\)
  \item Wanneer \(x\) al bestaat, dan wordt de \(x\) uit de \kw{var} in scope gebracht. Dat betekend dat de \(s\:x = y\)
  een andere \(x\), zeg \(x_1\), is dan de \(x\), zeg \(x_2\), uit \(s'\:x = z\). Deze \(x_2\) verdwijnt wanneer de
  functie uit scope gaat, zeg in state \(s''\). Na het uit scope gaan van \(x_2\) krijgen we weer \(s''\:x_1 = y\)
\end{itemize}

\Subsubsection{$D_F$ en func}
Zoals gesteld in~\ref{sec:Functies} wordt bij het opstarten van het programma de functie \functielabel{main} uitgevoerd
en worden alle functielabels globaal gemaakt. Om dit te kunnen doen moeten de functies ingelezen door middel van de
functie \kw{func}.

\(\left[\kw{func}^1\right]\) zorgt ervoor dat de functie gelezen wordt, en de functielabel globaal samen met zijn
defaultargumente worden toegevoegd aan de set met functielabels. \(\left[\kw{func}^2\right]\) zorgt ervoor dat als er
geen functiedeclaraties meer gevonden worden er met de \(main\) begonnen wordt. En deze functie
\(\left[\kw{func}^2\right]\) lost dat op.
\begin{eqnarray*}
\left[\kw{func}^1\right]&&
\langle\kw{func}\:fl\:tl\_callee\:tl\_caller\:vn\:D_V\:\kw{is}\:\kw{start}\:S\:\kw{end}
\:;\:D_F,\:s\rangle =>
s%
\end{eqnarray*}
\begin{flushright}
Waar we toepassen \(\kw{GLOBFL}(\kw{func}\:fl\:D_V\:S \:;\:D_F)\)
\end{flushright}

\begin{eqnarray*}
\left[\kw{func}^2\right]&&
\langle\epsilon,\:s\rangle =>
\langle\kw{spawn}\:fl:=main \:tl\_callee:=main \:tl\_caller:=\kw{undef}\:vn:=\kw{undef}\:tid=\kw{undef},\:s\rangle
\end{eqnarray*}

\Subsubsection{call}

FUNC\(_\mathrm{F}\) haalt door middel van een aanroep aan de executieomgeving de statements op die de
functie-implementatie vormen, van de functie die globaal bekend is onder de naam van het argument, de functielabel.
\[
	\mathrm{FUNC}_\mathrm{F} = \mathrm{FunctionLabel} \hookrightarrow \mathrm{Stm}
\]

\begin{eqnarray*}
\left[\kw{call}\right]&&
\osstep{}{\kw{call}\:fl,\:s}{S,\:s}\:\text{ Waar \(\mathrm{FUNC}_\mathrm{F}\:fl = \{S, D_V\}\)}
\end{eqnarray*}

\Subsubsection{spawn}
\begin{eqnarray*}
\left[\kw{spawn}\right]&&
\langle\kw{spawn}\:fl\:tl\_callee\:tl\_caller\:vn\:tid\:D_V,\:s\rangle =>
s\\
&&\parallel\langle D_V'\:;\:D_V\:;
\:\kw{call}\:fl,\:s[\kw{tl}\mapsto\mathcal{T}|[tl\_callee|]s][\kw{tid}\mapsto\mathcal{T\!L}|[tid|]s][\kw{vn}\mapsto\mathcal{V\!N}|[vn|]s]\rangle
\end{eqnarray*}
\begin{flushright}
Waar we toepassen \(\kw{GLOBTL}(tl)\) en \(\kw{LOCTL}(tl)\:s\) op \(\gamma\) van
\(\gamma\parallel\gamma'\) en waar \(\mathrm{FUNC}_\mathrm{F}\:fl = \{S, D_V'\}\)
\end{flushright}
\(vn\) moet de naam zijn van een lokale variabele in de caller

\Subsubsection{transaction}

In het begin van transactie wordt de state waarmee de transactie begint gekopieerd. Dit gebeurt bij het treffen van het
woord \kw{start_transaction}.
\begin{eqnarray*}
\left[\kw{transaction}\right]
&&\langle\kw{start_transaction}\:tid\:S\\
&&\kw{if}\:\kw{collect_votes}\:tid\:\kw{then}\\
&&\qquad S'\:\kw{if}\:b\:\kw{then}\\
&&\qquad\qquad S''\:\kw{commit_transaction}\:tid\\
&&\qquad\kw{else}\\
&&\qquad\qquad S'''\:\kw{rollback_transaction}\:tid\\
&&\kw{else}\\
&&\qquad S''''\:\kw{rollback_transaction}\:tid\\
&&,s\rangle
=> \\%
&&\langle S\:;\:\kw{if}\:\kw{collect_votes}\:tid\:\kw{then}\\
&&\qquad S'\:\kw{if}\:b\:\kw{then}\\
&&\qquad\qquad S''\:\kw{commit_transaction}\:tid\\
&&\qquad\kw{else}\\
&&\qquad\qquad S'''\:\kw{rollback_transaction}\:tid\\
&&\kw{else}\\
&&\qquad S''''\:\kw{rollback_transaction}\:tid\\
&&,s[\kw{tid}\mapsto\mathcal{T\!L}|[tid|]\rangle
\end{eqnarray*}
\begin{flushright}
Waar we toepassen \(\kw{GLOBTID}(tid)\) en \(\kw{GLOBSTATE}(tl\:tid)\)
\end{flushright}

\Subsubsection{commit}
\begin{eqnarray*}
\left[\kw{commit}\right]&&
\langle\kw{commit_transaction}\:tid,\:s\rangle => s\\
\end{eqnarray*}
\begin{flushright}
Waar we toepassen \kw{GLOBTID}(\(tid\))
\end{flushright}

\Subsubsection{rollback}
\begin{eqnarray*}
\left[\kw{rollback}\right]&&
\langle\kw{rollback_transaction}\:tid,\:s\rangle => s'\\
\end{eqnarray*}
\begin{flushright}
\(s'\) is waarde verkregen van \kw{GLOBSTATE''} met \(tid\) en \(tl\) uit \(s\). Waar we toepassen
\kw{GLOBTID'}(\(tid\))
\end{flushright}

\Subsubsection{end_transaction}
Als een subtaak met behulp van set_result tid het resultaat wil zetten

\begin{eqnarray*}
\left[\kw{end\_transaction}^\kw{tt}\right]
&& \langle\kw{if}\:\kw{set_result}\:tl\:vn\:a\:b\:tid\:\kw{then} \\
&& \qquad S_1\:\kw{end_commit_transaction}\:tl\:tid  \\
&& \kw{else} \\
&& \qquad S_2\:\kw{end_rollback_transaction}\:tl\:tid \\
&&,\:s\rangle
=> \\%
&&\langle S_1\:\kw{end_commit_transaction}\:tl\:tid, s\rangle
\end{eqnarray*}

\begin{eqnarray*}
\left[\kw{end\_transaction}^\kw{ff}\right]
&& \langle\kw{if}\:\kw{set_result}\:tl\:vn\:a\:b\:tid\:\kw{then} \\
&& \qquad S_1\:\kw{end_commit_transaction}\:tl\:tid  \\
&& \kw{else} \\
&& \qquad S_2\:\kw{end_rollback_transaction}\:tl\:tid \\
&&,\:s\rangle
=> \\%
&&\langle S_2\:\kw{end_rollback_transaction}\:tl\:tid, s\rangle
\end{eqnarray*}

\Subsubsection{end_commit}
\begin{eqnarray*}
\left[\kw{end_commit}\right]&&
\langle\kw{end_commit_transaction}\:tl\:tid,\:s\rangle => s\\
\end{eqnarray*}
\begin{flushright}
Waar we toepassen \kw{GLOBTID'}(\(tid\)) en \kw{LOCTID'}(\(tid\))
\end{flushright}

\Subsubsection{end_rollback}
\begin{eqnarray*}
\left[\kw{end_rollback}\right]&&
\langle\kw{end_rollback_transaction}\:tl\:tid,\:s\rangle => s'\\
\end{eqnarray*}
\begin{flushright}
\(s'\) is waarde verkregen van \kw{GLOBSTATE''} met \(tid\) en \(tl\) uit \(s\). Waar we toepassen
\kw{GLOBTID'}(\(tid\)) en \kw{LOCTID'}(\(tid\))
\end{flushright}
