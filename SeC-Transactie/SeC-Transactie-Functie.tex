\Section{Functies}

We breiden While uit met de notie van functies. \Index[functie]{Functies} zijn vergelijkbaar met de procedures
binnen While zoals beschreven in \cite[p. 52]{SemanticsWithApplications}. De definitie van functies is analoog aan die
van procedures in While. Bij het opstarten van een programma zal de executieomgeving alle functielabels
globaal zichtbaar maken. De volgorde waarin de functies worden gedefinieerd is niet van belang: voordat er een functie
wordt uitgevoerd zijn alle functies zichtbaar.

De executieomgeving zal bij het opstarten van een While programma de taak \functielabel{\Index{main}} opstarten met als
startfunctie de functie met het functielabel \functielabel{main} en een lege state.

Een programma wordt be\"eindigd als het einde van de \functielabel{main}-functie wordt bereikt of als geen enkele taak
meer loopt.

Om functies te kunnen voorzien van een naam, breiden we de meta-variabelen en syntactische categorie\"en uit:
\begin{equation*}
\begin{array}{rl}
	fl & \textrm{geeft een \index{functielabel}functielabel aan, \syncat{FunctionLabel}} \\
\end{array}
\end{equation*}

Een functie wordt gekenmerkt door de functienaam, de argumenten en de statements die de functie uitvoert. De
functienaam en de argumenten van de functie worden ook wel functietype of functiedeclaratie genoemd. De statements
binnen de functie worden ook wel functiebody of functie-implementatie genoemd.

De functie die aanroept wordt caller genoemd, en de functie die aangeroepen wordt, wordt callee genoemd.

\begin{functie}{func $fl$ $tl\_callee$ $tl\_caller$ $vn$ $tid$ $D_V$ \kw{is start} $S$ \kw{end}}
	\functieargument{$fl$: FunctieLabel, naam van de functie}
	\functieargument{$tl\_callee$: TaakLabel van de callee}
	\functieargument{$tl\_caller$: TaakLabel van de caller}
	\functieargument{$vn$: Naam van de result variabele in de caller}
	\functieargument{$tid$: TransactionLabel}
	Geeft aan van welke transactionele omgeving de callee onderdeel uitmaakt. Dit wordt gebruikt door
	\functienaam{collect_votes}. Als $tid$ \kw{undef} is, dan is de taak geen onderdeel van een transactie
	\functieargument{$D_V$}: Lijst met functieargumenten
	\functiebody{$S$: Statements van de functie-implementatie}
\end{functie}

\Subsection{Functieargumenten}
Argumenten worden doorgegeven aan de functies door een lijst met functieargumenten $D_V$. De exacte lijst van
functieargumenten wordt gedefinieerd door de functie zelf.

Een aantal functieargumenten zijn verplicht en staan als zodanig apart in de lijst met argumenten vermeld. Het is aan
de programmeur om deze variabelen van een juiste waarde te voorzien.

Alle \Index{functieargument}en worden ``by value'' doorgegeven. Het is mogelijk om in de functiedeclaratie de argument
van een default value te voorzien. In dat geval is het niet nodig voor om bij het aanroepen van de functie de waarden
van de variabelen mee te geven, maar zullen de variabelen in de functiebody zelf de default waarde hebben.

De argumenten worden ``by name'' doorgegeven. Het is niet relevant in welke volgorde de variabelen worden gedeclareerd of
meegegeven bij aanroep van een functie. Dat betekent wel dat de namen van de mee te geven variabelen uniek moeten zijn
binnen de aanroep.

De scope van de namen van de argumenten begint vanaf \kw{spawn} en eindigt bij het uitvoeren van de laatste statement van de
functie. Het is mogelijk om in een \kw{spawn} een argument variabele met de naam \(x\) te initialiseren met een
gelijknamige lokale variabele: \(var\:x:=x\).


