\Section{Parallellisme}

Een While programma bestaat uit een verzameling van taken (minimaal 1) die parallel aan elkaar worden uitgevoerd. Elke
taak voert een functie uit. Ieder While programma heeft tenminste \'e\'en functie --- met de naam \functielabel{main} ---
die door de executie-omgeving als eerste zal worden uitgevoerd. Iedere taak heeft een eigen state. Het is niet mogelijk
om direct, zonder tussenkomst van de executieomgeving, te communiceren tussen taken.

Om taken te kunnen voorzien van een naam, breiden we de meta-variabelen en syntactische categorie\"en uit:
\begin{equation*}
\begin{array}{rl}
	tl & \textrm{geeft een \Index{taaklabel} aan, \syncat{TaskLabel}}
\end{array}
\end{equation*}

\Subsection{spawn}
Om een taak op te starten introduceren we de statement \kw{\Index{spawn}}; door \kw{spawn} aan te roepen wordt een taak
gecre\"eerd die begint met het uitvoeren van het eerste statement van de functie die meegeven wordt.

De variabelen \(tl_caller\), \(tl_callee\), \(fl\), \(vn\) en \(tid\) zijn verplichte argumenten. De vraag of \(D_V\) als
argument meegegeven moet hangt af van de functie met functielabel \(fl\). De waarde van de diverse argumenten is sterk
gerelateerd aan de vereisten van de functiedeclaratie van de uit te voeren functie. Wanneer er geen argumenten meegegeven
worden, worden de defaultargumenten van de functie gebruikt.

Zoals aangegeven bij \kw{func} krijgt elke taak een variabele tot zijn beschikking in de caller. In
\kw{spawn} wordt de naam van deze variabele gezet. Deze variabele is alleen te benaderen met de primitieve functie
\kw{set_result}. Totdat \kw{set_result} is aangeroepen voor deze variabele is de waarde van deze variabele \kw{undef}.

Als er sprake is van van een transactionele omgeving alswel een niet-transactionele omgeving, dan moet \(tid\) een
geldige waarde hebben. Als er van een transactionele omgeving geen sprake is, dan krijgt \(tid\) de waarde \kw{undef}.
Als \(tid\) (transactielabel, zie \ref{sec:Transacties}) gedefinieerd is dan wordt de startstate van de taak opgeslagen.

\begin{tikzpicture}
\setcounter{threadnum}{1}
\stepcounter{threadnum}

% Instance 2
\draw[dotted] (0,      \thethreadnum*2-1)
		-- (\twidth,\thethreadnum*2-1)
		node [midway, right=.8\unitlength, below=0.75em] {};
\path (0,1)+(0,2) node[inststyle] (inst3) {\(T_2\)};
\draw[fill=gray!30] (4,\thethreadnum*2-1+\upperhalfbar)
				-- (4,\thethreadnum*2-1+\lowerhalfbar)
				-- (.5\twidth-.4\unitlength,\thethreadnum*2-1+\lowerhalfbar)
				--	(.5\twidth-.4\unitlength,\thethreadnum*2-1+\upperhalfbar)-- cycle;
\draw[fill=gray!30] (11,\thethreadnum*2-1+\upperhalfbar) % transactional env
				-- (11,\thethreadnum*2-1+\lowerhalfbar)
				-- (\twidth-1.4\unitlength,\thethreadnum*2-1+\lowerhalfbar)
				--	(\twidth-1.4\unitlength,\thethreadnum*2-1+\upperhalfbar)-- cycle;

\draw[] (9.5,\thethreadnum*2-1+2.75) % transactional env
				-- (9.5,\thethreadnum*2-1-.5)
				-- (\twidth-.9\unitlength,\thethreadnum*2-1-.5)
				--	(\twidth-.9\unitlength,\thethreadnum*2-1+2.75)-- cycle;

\node at (\twidth-.9\unitlength,\thethreadnum*2-1+2.75) [below left]{$\mathbf{tid}$};

\stepcounter{threadnum}

% Thread 1
\draw[dotted] (0,      \thethreadnum*2-1)
		-- (\twidth,\thethreadnum*2-1)
		node [midway, right=.8\unitlength, above=0.75em] {};
\path (0,2)+(0,3) node[inststyle] (inst2) {\(T_1\)};
\draw[fill=gray!30] (2,\thethreadnum*2-1+\upperhalfbar)
				-- (2,\thethreadnum*2-1+\lowerhalfbar)
				-- (.5\twidth-.4\unitlength,\thethreadnum*2-1+\lowerhalfbar)
				-- (.5\twidth-.4\unitlength,\thethreadnum*2-1+\upperhalfbar)-- cycle;
\draw[fill=gray!30] (2+.55\twidth,\thethreadnum*2-1+\upperhalfbar) % transactional env
				-- (2+.55\twidth,\thethreadnum*2-1+\lowerhalfbar)
				-- (\twidth-.4\unitlength,\thethreadnum*2-1+\lowerhalfbar)
				-- (\twidth-.4\unitlength,\thethreadnum*2-1+\upperhalfbar)-- cycle;

\stepcounter{threadnum}

% Result/argument arrows
\node (b) at (3.5,5) {};
\node (m) at (3.5,3) {};
\node (e) at (4.1,3) {};
\draw[->,>=angle 60] (b) |- (m) -- (e) node {};

\node (b) at (10.5,5) {}; \node (m) at (10.5,3) {}; \node (e) at (11.1,3) {}; \draw[->,>=angle 60] (b) |- (m) -- (e) node
{};
\end{tikzpicture}
Een schematische weergave van een \kw{spawn}. Links wordt een taak opgestart zonder gebruik te maken van een
transactionele omgeving, rechts wordt er wel gebruik van gemaakt.

\begin{functie}{spawn $fl$ $tl\_callee$ $tl\_caller$ $vn$ $tid$ $D_V$}
	\functieargument{$fl$: FunctieLabel van de functie}
	\functieargument{$tl\_callee$: TaakLabel van de callee}
	\functieargument{$tl\_caller$: TaakLabel van de caller}
	\functieargument{$vn$: Naam van de result variabele in de caller}
	\functieargument{$tid$: TransactionLabel}
	Geeft aan van welke transactionele omgeving de callee onderdeel uitmaakt. Dit wordt gebruikt door
	\functienaam{collect_votes}. Als $tid$ \kw{undef} is, dan is de taak geen onderdeel van een transactie
	\functieargument{$D_V$: De lijst met functieargumenten}
	\functieomschrijving
	Start een taak met het meegegeven label. Reserveer variabele $vn$ voor de return value van de callee
\end{functie}


\Subsection[Setresult]{set_result}
Om het resultaat van een taak terug te geven aan de caller introduceren we de (primitieve) booleaanse expressie
\kw{set_result}. De callee geeft met behulp van \kw{set_result} de waarde door aan de caller. Omdat de beide
taken van elkaar zijn afgeschermd zijn, gebeurt dit door tussenkomst van de executieomgeving.

Omdat niet altijd uit het domein van \textsl{waarde} op te maken is, of de berekening goed gegaan is, wordt
dit expliciet door de boolean \textsl{statusTaak} aangegeven.

Er zijn twee versies van \kw{set_result}. De ene versie is voor gebruik in een niet-transactionele omgeving. De andere
is voor gebruik in een transactionele omgeving en heeft een extra argument $tid$.

Als \kw{set_result} wordt aangeroepen in een transactionele omgeving, dan zal \kw{set_result} blokkeren totdat er
\kw{commit_transaction} of \kw{rollback_transaction} plaats vindt. Als \kw{set_result} wordt aangeroepen in dezelfde taak
als waarin ook \kw{start_transactie} is aangeroepen, dan blokkeert de \kw{set_result} nooit en returnt met dezelfde
waarde als de meegegeven statusTaak. Een (potentieel) blokkerende \kw{set_result} retourneert met \kw{true} bij een
\kw{commit_transaction} en \kw{false} bij een \kw{rollback_transaction}.

Alle taken die in het uitvoeren van \kw{wait} geblokkeerd zijn, worden onmiddelijk na een \kw{set_result}
gedeblokkeerd. De \kw{set_result} en \kw{wait} co\"ordinatie wordt atomair afgehandeld waarbij racecondities worden voorkomen.

Als \textsl{statusTaak} \kw{false} is, dan wordt is \(a\) door de executieomgeving op \kw{undef} gezet, ongeacht de
concrete waarde die bij de functieaanroep wordt doorgegeven.

\begin{functie}{set_result $tl$ $vn$ $a$ $b$}
	\functieargument{$tl$: Taaklabel van de caller}
	\functieargument{$vn$: Naam van de result variabele in de caller}
	\functieargument{$a$: Waarde van result variabele}
	\functieargument{$b$: StatusTaak}
	\functieresultaat{} \kw{true} als $b$ is true, \kw{false} als $b$ is false.
	\functieomschrijving
	Zet in de caller op de gereserveerde plaats \(vn\) de waarde \(a\)

	De waarde $b$ wordt gebruikt als return value van een eventuele \kw{wait}

	Als de caller een \kw{wait} gedaan heeft alvorens de callee een \kw{set_result} gedaan heeft, dan wordt de \kw{wait}
	van de caller geblokkeerd. Na het uitvoeren van een \kw{set_result} worden eventuele blokkades opgeheven
\end{functie}

\begin{functie}{set_result $tl$ $vn$ $a$ $b$ $tid$}
	\functieargument{$tl$: Taaklabel van de caller}
	\functieargument{$vn$: Naam van de result variabele in de caller}
	\functieargument{$a$: Waarde van result variabele}
	\functieargument{$b$: StatusTaak}
	\functieargument{$tid$: TransactionLabel}
	\functieresultaat{} \kw{true} als $tid$ en commit. \kw{false} als $tid$ en rollback.
	\functieomschrijving
	Zet in de caller op de gereserveerde plaats \(vn\) de waarde \(a\)

	De waarde $b$ wordt gebruikt als return value van een eventuele \kw{wait}

	Als de caller een \kw{wait} gedaan heeft alvorens de callee een \kw{set_result} gedaan heeft, dan wordt de \kw{wait}
	van de caller geblokkeerd. Na het uitvoeren van een \kw{set_result} worden eventuele blokkades opgeheven
\end{functie}

\Subsection{wait}
Om te wachten op het resultaat van een taak introduceren we de booleanse expressie \kw{wait}; met behulp van \kw{wait}
wacht de caller op het uitvoeren van \kw{set_result} door de callee.

\kw{wait} blokkeert zolang \kw{set_result} niet aangeroepen is door de callee. Als de callee \kw{set_result} heeft
aangeroepen voordat \kw{wait} wordt aangeroepen, zal \kw{wait} niet blokkeren. De executieomgeving zal er voor
zorgen dat hierbij geen racecondities op kunnen treden.

\begin{itemize}
\item Als \kw{set_result} is aangeroepen retourneert \kw{wait} \textsl{statusTaak}
\item Als de taak wordt be\"eindigd voordat \kw{set_result} wordt aangeroepen zal de executieomgeving \kw{false}
retourneren
\end{itemize}

Merk op dat de \kw{wait} de enige methode is voor de caller om te achterhalen wat het resultaat van een callee is.

\begin{functie}{wait $tl$}
	\functieargument{$tl$: Taaklabel van de callee}
	\functieresultaat{} \kw{true} als $b$ van \kw{set_result} \kw{true} is. \kw{false} anders.
	\functieomschrijving
	Een semaphore die de taak --- de caller van \kw{wait} --- blokkeert totdat er een resultaat bekend is voor de taak
	$tl$
\end{functie}

